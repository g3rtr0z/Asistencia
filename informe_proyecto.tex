\documentclass[12pt,a4paper]{article}

% Paquetes necesarios
\usepackage[utf8]{inputenc}
\usepackage[spanish]{babel}
\usepackage{graphicx}
\usepackage{hyperref}
\usepackage{listings}
\usepackage{xcolor}
\usepackage{geometry}
\usepackage{fancyhdr}
\usepackage{titlesec}
\usepackage{amsmath}
\usepackage{enumitem}
\usepackage{float}
\usepackage{booktabs}
\usepackage{multirow}
\usepackage{array}

% Configuración de página
\geometry{
    a4paper,
    left=3cm,
    right=2.5cm,
    top=3cm,
    bottom=2.5cm
}

% Configuración de encabezados y pies de página
\pagestyle{fancy}
\fancyhf{}
\fancyhead[L]{\leftmark}
\fancyhead[R]{\thepage}
\fancyfoot[C]{Mi Asistencia - Sistema de Gestión de Asistencia para Eventos}

% Configuración de código fuente
\lstset{
    language=JavaScript,
    basicstyle=\ttfamily\small,
    keywordstyle=\color{blue}\bfseries,
    commentstyle=\color{green!60!black},
    stringstyle=\color{red},
    numbers=left,
    numberstyle=\tiny\color{gray},
    stepnumber=1,
    numbersep=5pt,
    breaklines=true,
    breakatwhitespace=true,
    tabsize=2,
    frame=single,
    captionpos=b
}

% Títulos personalizados
\titleformat{\section}
{\Large\bfseries\color{blue!70!black}}
{\thesection}{1em}{}

\titleformat{\subsection}
{\large\bfseries\color{blue!50!black}}
{\thesubsection}{1em}{}

% Información del documento
\title{Sistema de Gestión de Asistencia para Eventos\\\large Mi Asistencia - Departamento de Informática\\\large Santo Tomás Temuco}
\author{Gerson Valdebenito}
\date{\today}

\begin{document}

% Portada
\begin{titlepage}
    \centering
    \vspace*{0.3cm}
    
    % Logo de la institución
    \includegraphics[width=0.25\textwidth]{src/assets/logopag.png}\\[1.2cm]
    
    % Institución
    {\large\textbf{INSTITUTO PROFESIONAL SANTO TOMÁS}}\\[0.3cm]
    {\large\textbf{DEPARTAMENTO DE INFORMÁTICA}}\\[0.3cm]
    {\large Sede Temuco}\\[2cm]
    
    % Línea decorativa
    \rule{\textwidth}{0.5pt}\\[1.5cm]
    
    % Título principal
    {\Huge\bfseries\color{blue!70!black} Sistema de Gestión de Asistencia\\para Eventos Académicos}\\[0.8cm]
    {\Large\itshape\color{blue!50!black} Mi Asistencia}\\[1.5cm]
    
    % Subtítulo
    {\large Solución Web para la Gestión Eficiente de Asistencia\\en Eventos del Departamento de Informática}\\[2.5cm]
    
    % Línea decorativa
    \rule{\textwidth}{0.5pt}\\[2cm]
    
    % Tipo de documento
    {\Large\bfseries PROYECTO DE TÍTULO}\\[2cm]
    
    % Información del estudiante
    {\large\textbf{Presentado por:}}\\[0.5cm]
    {\LARGE\bfseries Gerson Valdebenito}\\[1.5cm]
    
    {\large\textbf{Para optar al título de:}}\\[0.5cm]
    {\large [Ingeniería en Informática / Analista Programador / etc.]}\\[2cm]
    
    % Información del profesor guía (opcional, descomentar si aplica)
    % {\large\textbf{Profesor Guía:}}\\[0.5cm]
    % {\large [Nombre del Profesor]}\\[1.5cm]
    
    % Información institucional
    \vfill
    {\large\textbf{Región de La Araucanía, Chile}}\\[0.5cm]
    {\large\textbf{\today}}
    
    \vspace*{0.5cm}
\end{titlepage}

% Página de resumen
\newpage
\section*{Resumen Ejecutivo}
\addcontentsline{toc}{section}{Resumen Ejecutivo}

Este proyecto presenta el desarrollo e implementación de \textbf{Mi Asistencia}, un sistema web moderno y responsivo diseñado para la gestión eficiente de asistencia en eventos del Departamento de Informática del Instituto Profesional Santo Tomás, sede Temuco.

El sistema permite la gestión integral de eventos, registrando la asistencia de alumnos y funcionarios mediante la identificación por RUT. La aplicación está desarrollada con tecnologías web modernas como React, Firebase Firestore y Tailwind CSS, ofreciendo una experiencia de usuario optimizada tanto para dispositivos móviles como de escritorio.

Las funcionalidades principales incluyen: registro de asistencia en tiempo real, gestión de eventos, administración de participantes, generación de estadísticas, exportación de datos a Excel y un panel de administración completo. El sistema utiliza Firebase como backend, permitiendo sincronización en tiempo real y almacenamiento seguro de datos.

\textbf{Palabras clave:} Gestión de asistencia, React, Firebase, Eventos, Sistema web, Tiempo real.

% Índice
\newpage
\tableofcontents
\listoffigures
\listoftables

% Introducción
\newpage
\section{Introducción}

\subsection{Contexto y Motivación}

La gestión eficiente de asistencia en eventos académicos e institucionales representa un desafío constante para las organizaciones educativas. El Departamento de Informática del Instituto Profesional Santo Tomás, sede Temuco, requiere un sistema que permita registrar y gestionar la asistencia de manera ágil, precisa y en tiempo real durante diversos eventos, seminarios y actividades.

Los métodos tradicionales de registro de asistencia, como listas en papel o hojas de cálculo, presentan limitaciones significativas en términos de eficiencia, precisión y capacidad de generar reportes inmediatos. Además, la necesidad de adaptarse a diferentes tipos de eventos, que pueden involucrar tanto alumnos como funcionarios, requiere un sistema flexible y escalable.

\subsection{Objetivos del Proyecto}

\subsubsection{Objetivo General}

Desarrollar e implementar un sistema web de gestión de asistencia para eventos que permita registrar, consultar y gestionar la asistencia de participantes de manera eficiente, proporcionando información en tiempo real y herramientas de análisis para el Departamento de Informática de Santo Tomás Temuco.

\subsubsection{Objetivos Específicos}

\begin{enumerate}
    \item Diseñar una interfaz de usuario intuitiva y responsiva que funcione en dispositivos móviles y de escritorio.
    \item Implementar un sistema de registro de asistencia mediante identificación por RUT.
    \item Desarrollar un módulo de gestión de eventos que permita crear, editar y activar eventos.
    \item Crear un panel de administración completo para la gestión de participantes y eventos.
    \item Implementar funcionalidades de exportación de datos a formatos Excel para análisis posterior.
    \item Integrar Firebase Firestore para almacenamiento de datos y sincronización en tiempo real.
    \item Desarrollar un sistema de estadísticas que proporcione información inmediata sobre la asistencia.
\end{enumerate}

\subsection{Alcance del Proyecto}

El sistema está diseñado específicamente para uso en el Departamento de Informática de Santo Tomás Temuco, abarcando:

\begin{itemize}
    \item Gestión de eventos académicos e institucionales
    \item Registro de asistencia de alumnos y funcionarios
    \item Generación de reportes y estadísticas
    \item Exportación de datos para análisis posterior
    \item Panel de administración con control de acceso
\end{itemize}

% Marco Teórico
\newpage
\section{Marco Teórico}

\subsection{Tecnologías Web Modernas}

\subsubsection{React}

React es una biblioteca de JavaScript desarrollada por Facebook para construir interfaces de usuario. Su arquitectura basada en componentes reutilizables permite desarrollar aplicaciones complejas de manera modular y mantenible \cite{react-docs}. 

Las características principales que hacen de React una tecnología adecuada para este proyecto incluyen:

\begin{itemize}
    \item \textbf{Componentes reutilizables:} Permite crear componentes que pueden ser utilizados en múltiples contextos.
    \item \textbf{Virtual DOM:} Optimiza el rendimiento al actualizar solo los componentes que han cambiado.
    \item \textbf{Ecosistema rico:} Gran cantidad de bibliotecas y herramientas complementarias.
    \item \textbf{Hooks:} Permiten manejar estado y efectos secundarios de manera funcional.
\end{itemize}

\subsubsection{Firebase}

Firebase es una plataforma de desarrollo de aplicaciones móviles y web de Google que proporciona servicios backend como base de datos, autenticación y almacenamiento \cite{firebase-docs}. Para este proyecto se utiliza principalmente:

\begin{itemize}
    \item \textbf{Firestore:} Base de datos NoSQL en tiempo real que permite sincronización automática.
    \item \textbf{Authentication:} Sistema de autenticación seguro para el panel de administración.
    \item \textbf{Real-time updates:} Actualizaciones en tiempo real sin necesidad de recargar la página.
\end{itemize}

\subsubsection{Tailwind CSS}

Tailwind CSS es un framework de CSS utility-first que permite construir diseños personalizados rápidamente mediante clases de utilidad \cite{tailwind-docs}. Sus ventajas incluyen:

\begin{itemize}
    \item Desarrollo rápido de interfaces responsivas
    \item Clases utilitarias que pueden combinarse de manera flexible
    \item Optimización automática del CSS final
    \item Diseño mobile-first por defecto
\end{itemize}

\subsection{Arquitectura de Aplicaciones Web Modernas}

Las aplicaciones web modernas siguen una arquitectura de cliente-servidor donde:

\begin{itemize}
    \item El \textbf{cliente} (navegador) maneja la presentación y la lógica de interfaz.
    \item El \textbf{servidor} (backend) gestiona la lógica de negocio y el almacenamiento de datos.
    \item La comunicación se realiza mediante APIs REST o servicios en tiempo real.
\end{itemize}

En este proyecto, Firebase actúa como Backend as a Service (BaaS), eliminando la necesidad de desarrollar y mantener un servidor propio.

\subsection{Sistemas de Gestión de Asistencia}

Los sistemas modernos de gestión de asistencia deben considerar:

\begin{enumerate}
    \item \textbf{Identificación única:} Sistema de identificación que garantice la unicidad de cada participante.
    \item \textbf{Registro en tiempo real:} Capacidad de registrar y visualizar asistencia instantáneamente.
    \item \textbf{Reportes y análisis:} Generación de reportes que permitan análisis posterior.
    \item \textbf{Escalabilidad:} Capacidad de manejar eventos con diferente número de participantes.
    \item \textbf{Accesibilidad:} Interfaz accesible desde múltiples dispositivos.
\end{enumerate}

% Metodología
\newpage
\section{Metodología}

\subsection{Metodología de Desarrollo}

El proyecto se desarrolló siguiendo una metodología ágil adaptada, combinando elementos de desarrollo iterativo e incremental. Esta aproximación permitió:

\begin{itemize}
    \item Desarrollo en iteraciones cortas con funcionalidades completas
    \item Feedback continuo durante el desarrollo
    \item Adaptabilidad a cambios en los requerimientos
    \item Entrega de valor incremental
\end{itemize}

\subsection{Fases del Proyecto}

\subsubsection{Fase 1: Análisis y Diseño}

En esta fase se realizó:

\begin{itemize}
    \item Análisis de requerimientos funcionales y no funcionales
    \item Diseño de la arquitectura del sistema
    \item Definición de casos de uso
    \item Diseño de la base de datos (estructura de colecciones Firestore)
    \item Diseño de la interfaz de usuario (wireframes y mockups)
\end{itemize}

\subsubsection{Fase 2: Configuración e Infraestructura}

\begin{itemize}
    \item Configuración del proyecto React con Vite
    \item Configuración de Firebase (Firestore y Authentication)
    \item Establecimiento de la estructura de carpetas del proyecto
    \item Configuración de herramientas de desarrollo (ESLint, Prettier)
    \item Implementación del sistema de estilos con Tailwind CSS
\end{itemize}

\subsubsection{Fase 3: Desarrollo de Componentes Base}

\begin{itemize}
    \item Desarrollo de componentes UI reutilizables (Button, Input, Card)
    \item Implementación del sistema de routing con React Router
    \item Creación de hooks personalizados para gestión de datos
    \item Desarrollo de servicios para comunicación con Firebase
\end{itemize}

\subsubsection{Fase 4: Funcionalidades Principales}

\begin{itemize}
    \item Desarrollo del módulo de registro de asistencia
    \item Implementación del panel de administración
    \item Creación del sistema de gestión de eventos
    \item Desarrollo de módulo de estadísticas
\end{itemize}

\subsubsection{Fase 5: Funcionalidades Avanzadas}

\begin{itemize}
    \item Sistema de importación de datos (Excel y JSON)
    \item Módulo de exportación a Excel
    \item Optimizaciones de rendimiento
    \item Mejoras en la experiencia de usuario
\end{itemize}

\subsubsection{Fase 6: Pruebas y Ajustes}

\begin{itemize}
    \item Pruebas funcionales de cada módulo
    \item Pruebas de usabilidad
    \item Corrección de errores
    \item Optimización final
\end{itemize}

\subsection{Herramientas de Desarrollo}

\begin{itemize}
    \item \textbf{Editor:} Visual Studio Code
    \item \textbf{Control de versiones:} Git
    \item \textbf{Gestor de paquetes:} npm
    \item \textbf{Build tool:} Vite
    \item \textbf{Linter:} ESLint
    \item \textbf{Formatter:} Prettier
\end{itemize}

% Desarrollo e Implementación
\newpage
\section{Desarrollo e Implementación}

\subsection{Arquitectura del Sistema}

El sistema está estructurado siguiendo una arquitectura de componentes React organizada de manera jerárquica. La estructura principal del proyecto es:

\begin{lstlisting}[caption=Estructura de carpetas del proyecto]
src/
├── components/          # Componentes React
│   ├── admin/         # Componentes de administración
│   ├── alumnos/       # Componentes de alumnos
│   ├── eventos/       # Componentes de eventos
│   ├── trabajadores/  # Componentes de funcionarios
│   └── ui/            # Componentes UI reutilizables
├── hooks/              # Custom hooks
├── services/           # Servicios de Firebase
├── utils/              # Utilidades y helpers
├── constants/          # Constantes del proyecto
├── connection/         # Configuración Firebase
└── assets/             # Recursos estáticos
\end{lstlisting}

\subsection{Componentes Principales}

\subsubsection{Componente Inicio (Registro de Asistencia)}

El componente \texttt{Inicio.jsx} es la interfaz principal para el registro de asistencia. Permite:

\begin{itemize}
    \item Ingreso de RUT del participante
    \item Búsqueda y validación del participante en la base de datos
    \item Registro de asistencia al evento activo
    \item Visualización de confirmación con datos del participante
\end{itemize}

\subsubsection{Panel de Administración}

El \texttt{AdminPanel.jsx} proporciona una interfaz completa para la administración del sistema, incluyendo:

\begin{itemize}
    \item Gestión de eventos (crear, editar, activar)
    \item Gestión de participantes (agregar, eliminar, importar)
    \item Visualización de listas y estadísticas
    \item Exportación de datos
\end{itemize}

\subsubsection{Servicios de Datos}

Los servicios implementados en \texttt{services/} manejan toda la comunicación con Firebase:

\begin{itemize}
    \item \texttt{alumnosService.js:} CRUD de alumnos y suscripciones en tiempo real
    \item \texttt{eventosService.js:} Gestión de eventos
    \item \texttt{eventosAlumnosService.js:} Relación eventos-alumnos
\end{itemize}

\subsection{Base de Datos}

La estructura de datos en Firestore sigue el siguiente esquema:

\begin{itemize}
    \item \textbf{Colección eventos:} Contiene los eventos con sus propiedades
    \item \textbf{Subcolección alumnos:} Dentro de cada evento, contiene los alumnos participantes
    \item \textbf{Colección usuarios:} Para autenticación de administradores
\end{itemize}

\subsection{Funcionalidades Implementadas}

\subsubsection{Registro de Asistencia}

El sistema permite registrar asistencia mediante:

\begin{enumerate}
    \item Ingreso del RUT del participante
    \item Validación del RUT en el evento activo
    \item Actualización del estado de presencia en tiempo real
    \item Confirmación visual inmediata
\end{enumerate}

\subsubsection{Gestión de Eventos}

Los administradores pueden:

\begin{itemize}
    \item Crear nuevos eventos con nombre, descripción, fechas y tipo
    \item Editar eventos existentes
    \item Activar/desactivar eventos
    \item Filtrar eventos por tipo (alumnos o funcionarios)
\end{itemize}

\subsubsection{Estadísticas en Tiempo Real}

El sistema proporciona estadísticas actualizadas automáticamente:

\begin{itemize}
    \item Total de participantes
    \item Número de presentes
    \item Número de ausentes
    \item Exportación de estadísticas filtradas a Excel
\end{itemize}

\subsubsection{Importación y Exportación de Datos}

\begin{itemize}
    \item Importación desde archivos Excel y JSON
    \item Exportación de listas completas y filtradas a Excel
    \item Validación de datos durante la importación
\end{itemize}

\subsection{Seguridad}

El sistema implementa seguridad mediante:

\begin{itemize}
    \item Autenticación mediante Firebase Authentication
    \item Control de acceso al panel de administración
    \item Validación de datos en cliente y servidor
    \item Reglas de seguridad de Firestore
\end{itemize}

% Resultados
\newpage
\section{Resultados}

\subsection{Interfaz de Usuario}

La interfaz desarrollada es:

\begin{itemize}
    \item \textbf{Responsive:} Funciona correctamente en dispositivos móviles, tablets y escritorio
    \item \textbf{Intuitiva:} Navegación clara y fácil de usar
    \item \textbf{Moderna:} Diseño actual siguiendo principios de Material Design
    \item \textbf{Accesible:} Cumple con estándares básicos de accesibilidad web
\end{itemize}

\subsection{Funcionalidades Implementadas}

Todas las funcionalidades planificadas fueron implementadas exitosamente:

\begin{itemize}
    \item Sistema de registro de asistencia operativo
    \item Panel de administración completo y funcional
    \item Gestión de eventos implementada
    \item Estadísticas en tiempo real funcionando
    \item Importación y exportación de datos operativa
    \item Sistema de autenticación implementado
\end{itemize}

\subsection{Rendimiento}

El sistema presenta:

\begin{itemize}
    \item Carga inicial rápida gracias a Vite
    \item Actualizaciones en tiempo real sin lag perceptible
    \item Interfaz fluida y responsiva
    \item Optimización de consultas a la base de datos
\end{itemize}

\subsection{Pruebas Realizadas}

Se realizaron pruebas exhaustivas de:

\begin{itemize}
    \item Funcionalidad de cada módulo
    \item Integración entre componentes
    \item Usabilidad en diferentes dispositivos
    \item Carga con diferentes volúmenes de datos
    \item Seguridad del sistema de autenticación
\end{itemize}

% Conclusiones
\newpage
\section{Conclusiones}

\subsection{Objetivos Cumplidos}

Todos los objetivos propuestos fueron alcanzados satisfactoriamente:

\begin{enumerate}
    \item Se desarrolló una interfaz intuitiva y responsiva que funciona en múltiples dispositivos
    \item Se implementó un sistema de registro de asistencia mediante RUT
    \item Se creó un módulo completo de gestión de eventos
    \item Se desarrolló un panel de administración con todas las funcionalidades necesarias
    \item Se implementó la exportación de datos a Excel
    \item Se integró Firebase Firestore exitosamente
    \item Se desarrolló un sistema de estadísticas en tiempo real
\end{enumerate}

\subsection{Aprendizajes}

Durante el desarrollo del proyecto se adquirieron conocimientos en:

\begin{itemize}
    \item Desarrollo de aplicaciones web modernas con React
    \item Integración con servicios backend como Firebase
    \item Diseño responsivo con Tailwind CSS
    \item Arquitectura de aplicaciones escalables
    \item Gestión de estado en aplicaciones React
    \item Optimización de rendimiento web
\end{itemize}

\subsection{Trabajo Futuro}

Para futuras mejoras se considera:

\begin{itemize}
    \item Implementación de notificaciones push
    \item Generación de códigos QR para registro rápido
    \item Integración con sistemas institucionales existentes
    \item Dashboard de análisis más avanzado
    \item Aplicación móvil nativa
    \item Sistema de reportes automáticos por email
\end{itemize}

\subsection{Impacto}

El sistema desarrollado proporciona:

\begin{itemize}
    \item Eficiencia en el registro de asistencia
    \item Reducción de errores humanos
    \item Acceso inmediato a estadísticas
    \item Mejor organización de eventos
    \item Experiencia de usuario mejorada
\end{itemize}

% Bibliografía
\newpage
\section{Referencias Bibliográficas}

\begin{thebibliography}{99}

\bibitem{react-docs}
Meta Platforms, Inc. (2024). \textit{React Documentation}. 
Disponible en: \url{https://react.dev}

\bibitem{firebase-docs}
Google LLC. (2024). \textit{Firebase Documentation}. 
Disponible en: \url{https://firebase.google.com/docs}

\bibitem{tailwind-docs}
Tailwind Labs, Inc. (2024). \textit{Tailwind CSS Documentation}. 
Disponible en: \url{https://tailwindcss.com/docs}

\bibitem{vite-docs}
Evan You. (2024). \textit{Vite Documentation}. 
Disponible en: \url{https://vitejs.dev}

\bibitem{javascript-mdn}
Mozilla Developer Network. (2024). \textit{JavaScript Guide}. 
Disponible en: \url{https://developer.mozilla.org/en-US/docs/Web/JavaScript}

\end{thebibliography}

% Anexos
\newpage
\appendix
\section{Anexo A: Estructura de Datos}

\subsection{Modelo de Datos Evento}

\begin{lstlisting}[caption=Estructura de un evento en Firestore]
{
  id: string,
  nombre: string,
  descripcion: string,
  fechaInicio: Timestamp,
  fechaFin: Timestamp,
  tipo: 'alumnos' | 'trabajadores',
  activo: boolean,
  capacidad: number,
  createdAt: Timestamp,
  updatedAt: Timestamp
}
\end{lstlisting}

\subsection{Modelo de Datos Alumno}

\begin{lstlisting}[caption=Estructura de un alumno en Firestore]
{
  id: string,
  rut: string,
  nombre: string,
  nombres: string,
  apellidos: string,
  carrera: string,
  institucion: string,
  grupo: string,
  asiento: string,
  presente: boolean,
  asiste: boolean,
  vegano: string,
  departamento: string,
  createdAt: Timestamp,
  updatedAt: Timestamp
}
\end{lstlisting}

\section{Anexo B: Capturas de Pantalla}

\textit{Nota: Se recomienda incluir capturas de pantalla de las principales interfaces del sistema en esta sección.}

\section{Anexo C: Manual de Usuario}

\textit{Nota: Se puede incluir un manual detallado de uso del sistema en esta sección.}

\end{document}

