\documentclass[12pt,a4paper]{article}

% Paquetes necesarios
\usepackage[utf8]{inputenc}
\usepackage[spanish]{babel}
\usepackage{graphicx}
\usepackage{hyperref}
\usepackage{xcolor}
\usepackage{geometry}
\usepackage{fancyhdr}
\usepackage{titlesec}
\usepackage{enumitem}
\usepackage{booktabs}
\usepackage{float}
\usepackage{longtable}
\usepackage{array}
\usepackage{multirow}

% Configuración de página
\geometry{
    a4paper,
    left=3cm,
    right=2.5cm,
    top=3cm,
    bottom=2.5cm
}

% Configuración de encabezados y pies de página
\pagestyle{fancy}
\fancyhf{}
\fancyhead[L]{\leftmark}
\fancyhead[R]{\thepage}
\fancyfoot[C]{Mi Asistencia - Informe de Proyecto de Título}

% Colores corporativos Santo Tomás
\definecolor{colorprincipal}{RGB}{0, 99, 65}
\definecolor{colorsecundario}{RGB}{0, 179, 136}

% Títulos personalizados
\titleformat{\section}
{\Large\bfseries\color{colorprincipal}}
{\thesection}{1em}{}

\titleformat{\subsection}
{\large\bfseries\color{colorsecundario}}
{\thesubsection}{1em}{}

% Información del documento
\title{Sistema de Gestión de Asistencia para Eventos\\\large Mi Asistencia\\\large Informe de Proyecto de Título}
\author{Gerson Valdebenito}
\date{\today}

\begin{document}

% ============================================
% PORTADA
% ============================================
\begin{titlepage}
    \centering
    \vspace*{0.5cm}
    
    % Logo de la institución
    \includegraphics[width=0.25\textwidth]{src/assets/logopag.png}\\[1cm]
    
    {\Huge\bfseries\color{colorprincipal} Sistema de Gestión de\\Asistencia para Eventos}\\[0.5cm]
    {\LARGE\itshape Mi Asistencia}\\[1.5cm]
    
    {\large Informe de Proyecto de Título}\\[2cm]
    
    {\large Instituto Profesional Santo Tomás}\\[0.5cm]
    {\large Departamento de Informática}\\[0.5cm]
    {\large Sede Temuco}\\[2cm]
    
    {\large\date\today}
    
    \vfill
    
    {\small Desarrollado por: Gerson Valdebenito}
\end{titlepage}

% ============================================
% ÍNDICE
% ============================================
\newpage
\tableofcontents
\newpage

% ============================================
% RESUMEN EJECUTIVO
% ============================================
\section*{Resumen Ejecutivo}
\addcontentsline{toc}{section}{Resumen Ejecutivo}

Este proyecto presenta el desarrollo e implementación de \textbf{Mi Asistencia}, un sistema web moderno y responsivo diseñado para la gestión eficiente de asistencia en eventos académicos y ceremonias de titulación del Departamento de Informática del Instituto Profesional Santo Tomás, sede Temuco.

El sistema surge como respuesta a los problemas identificados en los métodos tradicionales de registro de asistencia, que consumen entre 3 y 4 horas de trabajo administrativo por evento, generan errores de transcripción del 11-24\%, y no proporcionan información en tiempo real durante el desarrollo de los eventos.

La solución propuesta es una aplicación web desarrollada con React y Firebase que permite registrar asistencia mediante identificación por RUT en menos de 10 segundos por persona, proporcionando estadísticas actualizadas automáticamente y exportación de reportes a Excel con un solo clic. El sistema funciona desde cualquier navegador web sin necesidad de instalaciones, siendo accesible desde dispositivos móviles, tablets y computadores.

Los objetivos principales del proyecto incluyen simplificar el proceso de registro, proporcionar información en tiempo real, reducir el tiempo administrativo en un 70-80\%, eliminar errores de transcripción, y facilitar el análisis posterior de datos. Los resultados esperados contemplan una aplicación completamente funcional, documentación técnica completa, manual de usuario, y un sistema listo para implementación inmediata.

La metodología utilizada fue un enfoque ágil adaptado, con desarrollo iterativo e incremental, permitiendo feedback continuo y adaptabilidad a cambios. El proyecto se desarrolló en fases que incluyeron análisis y diseño, configuración de infraestructura, desarrollo de componentes base, funcionalidades principales, funcionalidades avanzadas, y pruebas finales.

El sistema ha sido completamente desarrollado y probado, demostrando capacidad para manejar eventos desde 10 hasta más de 1000 participantes con la misma eficiencia. Los resultados preliminares indican una reducción del 70-80\% en tiempo administrativo y eliminación prácticamente total de errores, cumpliendo con todos los objetivos propuestos.

\vspace{0.5cm}

\textbf{Palabras clave:} Gestión de asistencia, React, Firebase, Eventos académicos, Sistema web, Tiempo real, Automatización

% ============================================
% SECCIÓN 1
% ============================================
\section{Introducción}

\subsection{Problemática}

La gestión eficiente de asistencia en eventos académicos representa un desafío constante para las instituciones educativas. En el Instituto Profesional Santo Tomás de Temuco, específicamente en el Departamento de Informática, esta situación se ha presentado de manera recurrente durante la organización de seminarios, talleres, ceremonias de titulación y otras actividades académicas.

El problema actual radica en que los métodos tradicionales utilizados para el registro de asistencia presentan múltiples limitaciones. Durante mucho tiempo se intentó gestionar todo mediante métodos manuales: listas impresas en papel, planillas de Excel que requerían actualización posterior al evento, y anotaciones que frecuentemente resultaban difíciles de interpretar. Los resultados obtenidos eran consistentemente problemáticos: información incompleta, errores frecuentes que afectaban entre el 11\% y 24\% de los registros, y la imposibilidad de determinar con certeza la cantidad real de participantes durante el desarrollo del evento.

Cuando los métodos manuales resultaban insuficientes, especialmente en eventos con más de 200 participantes, la alternativa más común era contratar servicios externos especializados. Si bien esta opción parecía viable inicialmente, presentaba inconvenientes significativos: cada evento requería un nuevo pago, y considerando que en una institución como Santo Tomás se realizan actividades académicas prácticamente todas las semanas, estos costos se acumulaban considerablemente a lo largo del año.

El impacto de este problema se manifiesta en múltiples dimensiones. Desde el punto de vista operativo, un evento típico con 200 participantes consume entre 3 y 4 horas de trabajo administrativo, incluyendo preparación de listas (30-60 minutos), registro durante el evento (2-3 horas), conteo manual (30 minutos), transcripción a Excel (1 hora), y generación de reportes (30 minutos adicionales). Esta carga de trabajo recae directamente sobre el personal administrativo, coordinadores y secretarias, quienes deben dedicar tiempo valioso a tareas repetitivas que podrían automatizarse.

Adicionalmente, la falta de información en tiempo real impide tomar decisiones informadas durante el desarrollo del evento. Los organizadores no pueden conocer cuántos participantes han llegado realmente hasta que se completa el conteo manual al final, lo que limita la capacidad de ajustar recursos, enviar recordatorios, o modificar el programa según la asistencia real.

La necesidad de una solución propia e institucional se hace evidente cuando se considera la dependencia de terceros, la falta de control sobre la información, y los costos acumulativos de contratar servicios externos. Una herramienta desarrollada específicamente para las necesidades del departamento, que esté permanentemente disponible y permita gestionar todos los aspectos desde una plataforma centralizada, no solo representaría un ahorro económico a largo plazo, sino que proporcionaría el control real de la información: saber exactamente qué está ocurriendo en el momento en que sucede, y poder tomar decisiones fundamentadas en datos concretos y actualizados.

La relevancia de este proyecto se justifica por su potencial impacto en la eficiencia operativa del departamento, la mejora en la calidad de la información generada, y la experiencia tanto de participantes como de administradores. Además, representa una oportunidad para aplicar conocimientos técnicos adquiridos durante la formación académica en un contexto real, resolviendo un problema concreto de la institución.

\subsection{Objetivos}

\subsubsection{Objetivo General}

Desarrollar e implementar un sistema web de gestión de asistencia para eventos académicos que permita registrar, consultar y gestionar la asistencia de participantes de manera eficiente, proporcionando información en tiempo real y herramientas de análisis para el Departamento de Informática de Santo Tomás Temuco.

\subsubsection{Objetivos Específicos}

\begin{enumerate}
    \item Diseñar una interfaz de usuario intuitiva y responsiva que funcione en dispositivos móviles y de escritorio, adaptándose a diferentes tamaños de pantalla y garantizando una experiencia de usuario óptima.
    
    \item Implementar un sistema de registro de asistencia mediante identificación por RUT que valide automáticamente los datos, prevenga duplicados, y complete el registro en menos de 10 segundos por persona.
    
    \item Desarrollar un módulo de gestión de eventos que permita crear, editar, activar y desactivar eventos, con capacidad para gestionar múltiples eventos de forma independiente.
    
    \item Crear un panel de administración completo para la gestión de participantes y eventos, incluyendo funciones de importación masiva desde Excel y JSON, y exportación de reportes en diferentes formatos.
    
    \item Implementar funcionalidades de exportación de datos a formatos Excel para análisis posterior, permitiendo generar reportes completos o filtrados (presentes, ausentes, total) con un solo clic.
    
    \item Integrar Firebase Firestore para almacenamiento de datos y sincronización en tiempo real, garantizando que la información se actualice automáticamente sin necesidad de recargar la página.
    
    \item Desarrollar un sistema de estadísticas que proporcione información inmediata sobre la asistencia, mostrando totales, presentes, ausentes y porcentajes actualizados automáticamente durante el evento.
    
    \item Implementar un sistema de autenticación seguro para el panel de administración, garantizando que solo usuarios autorizados puedan acceder a funciones administrativas.
    
    \item Realizar pruebas exhaustivas del sistema para validar su funcionamiento con diferentes volúmenes de datos, desde eventos pequeños (10 participantes) hasta eventos grandes (más de 1000 participantes).
    
    \item Documentar completamente el sistema, incluyendo documentación técnica, manual de usuario, y guías de implementación para facilitar su uso y mantenimiento futuro.
\end{enumerate}

\subsection{Resultados Esperados}

\subsubsection{Resultado 1: Aplicación Web Funcional}

\textbf{Descripción del entregable:} Una aplicación web completamente funcional desarrollada con React y Firebase, accesible desde cualquier navegador moderno, que permite gestionar la asistencia en eventos académicos de manera eficiente.

\textbf{Criterios de aceptación:}
\begin{itemize}
    \item La aplicación debe cargar correctamente en navegadores modernos (Chrome, Firefox, Safari, Edge)
    \item Debe funcionar en dispositivos móviles, tablets y computadores
    \item El registro de asistencia debe completarse en menos de 10 segundos por persona
    \item Las estadísticas deben actualizarse automáticamente sin recargar la página
    \item El sistema debe manejar eventos con al menos 1000 participantes sin degradación de rendimiento
\end{itemize}

\textbf{Impacto esperado:} Transformación del proceso de gestión de asistencia de manual a digital, reduciendo el tiempo de trabajo administrativo y mejorando la precisión de los registros.

\textbf{Métricas de éxito:}
\begin{itemize}
    \item Tiempo de registro: menos de 10 segundos por persona
    \item Tasa de errores: menos del 1\%
    \item Tiempo de carga inicial: menos de 3 segundos
    \item Disponibilidad del sistema: 99\% o superior
\end{itemize}

\subsubsection{Resultado 2: Reducción del Tiempo Administrativo}

\textbf{Descripción del entregable:} Reducción medible del tiempo invertido en gestión de asistencia, pasando de 4-5 horas a 45 minutos por evento típico con 200 participantes.

\textbf{Criterios de aceptación:}
\begin{itemize}
    \item Configuración inicial de evento: máximo 15 minutos
    \item Monitoreo durante evento: mínimo tiempo de atención activa
    \item Exportación de reportes: máximo 1 minuto
    \item Reducción total del tiempo: al menos 70\% comparado con método tradicional
\end{itemize}

\textbf{Impacto esperado:} Liberación de recursos humanos para actividades de mayor valor estratégico, reducción de carga administrativa del personal, y mejora en la eficiencia operativa del departamento.

\textbf{Métricas de éxito:}
\begin{itemize}
    \item Ahorro de tiempo: 70-80\% por evento
    \item Tiempo total de gestión: máximo 45 minutos por evento
    \item Tiempo de configuración: máximo 15 minutos
    \item Tiempo de exportación: máximo 1 minuto
\end{itemize}

\subsubsection{Resultado 3: Eliminación de Errores de Transcripción}

\textbf{Descripción del entregable:} Sistema que valida automáticamente todos los datos ingresados, previene duplicados, y elimina prácticamente todos los errores asociados al proceso manual.

\textbf{Criterios de aceptación:}
\begin{itemize}
    \item Validación automática de RUT antes de aceptar registro
    \item Prevención de registros duplicados con mensaje claro al usuario
    \item Almacenamiento automático e instantáneo de todos los datos
    \item Conteo automático de estadísticas sin intervención manual
    \item Tasa de errores: menos del 1\%
\end{itemize}

\textbf{Impacto esperado:} Información precisa y confiable disponible inmediatamente, eliminación de costos asociados a corrección de errores, y mejora en la calidad de los datos para toma de decisiones.

\textbf{Métricas de éxito:}
\begin{itemize}
    \item Reducción de errores: del 11-24\% a menos del 1\%
    \item Precisión de datos: 99\% o superior
    \item Validación de RUT: 100\% de los registros validados
    \item Prevención de duplicados: 100\% efectiva
\end{itemize}

\subsubsection{Resultado 4: Información en Tiempo Real}

\textbf{Descripción del entregable:} Sistema de estadísticas que se actualiza automáticamente durante el evento, proporcionando información inmediata sobre asistencia sin necesidad de recargar la página.

\textbf{Criterios de aceptación:}
\begin{itemize}
    \item Estadísticas visibles en tiempo real (totales, presentes, ausentes, porcentajes)
    \item Actualización automática sin recargar página
    \item Tiempo de actualización: menos de 1 segundo después de cada registro
    \item Disponibilidad de información durante todo el evento
\end{itemize}

\textbf{Impacto esperado:} Capacidad de tomar decisiones informadas durante el desarrollo del evento, optimización de recursos según asistencia real, y mejora en la planificación y ejecución de eventos.

\textbf{Métricas de éxito:}
\begin{itemize}
    \item Tiempo de actualización: menos de 1 segundo
    \item Disponibilidad de estadísticas: 100\% del tiempo durante el evento
    \item Precisión de estadísticas: 100\%
    \item Facilidad de acceso: disponible desde cualquier dispositivo
\end{itemize}

\subsubsection{Resultado 5: Documentación Completa}

\textbf{Descripción del entregable:} Documentación técnica completa, manual de usuario detallado, y guías de implementación que faciliten el uso y mantenimiento del sistema.

\textbf{Criterios de aceptación:}
\begin{itemize}
    \item Documentación técnica con arquitectura del sistema
    \item Manual de usuario con capturas de pantalla y guías paso a paso
    \item Guía de implementación con requisitos y proceso de instalación
    \item Documentación de API y servicios utilizados
\end{itemize}

\textbf{Impacto esperado:} Facilita la adopción del sistema por parte de los usuarios, permite mantenimiento futuro, y asegura la transferencia de conocimiento.

\textbf{Métricas de éxito:}
\begin{itemize}
    \item Cobertura de documentación: 100\% de las funcionalidades documentadas
    \item Claridad: usuarios pueden usar el sistema sin capacitación previa usando solo el manual
    \item Completitud: todas las secciones del manual completadas
\end{itemize}

\subsection{Metodología}

\subsubsection{Enfoque Metodológico}

El proyecto se desarrolló siguiendo una metodología ágil adaptada, combinando elementos de desarrollo iterativo e incremental. Esta aproximación permitió desarrollo en iteraciones cortas con funcionalidades completas, feedback continuo durante el desarrollo, adaptabilidad a cambios en los requerimientos, y entrega de valor incremental.

\subsubsection{Fases del Proyecto}

El desarrollo se estructuró en las siguientes fases principales:

\textbf{Fase 1: Análisis y Diseño}
\begin{itemize}
    \item Análisis de requerimientos funcionales y no funcionales
    \item Diseño de la arquitectura del sistema
    \item Definición de casos de uso
    \item Diseño de la base de datos (estructura de colecciones Firestore)
    \item Diseño de la interfaz de usuario (wireframes y mockups)
\end{itemize}

\textbf{Fase 2: Configuración e Infraestructura}
\begin{itemize}
    \item Configuración del proyecto React con Vite
    \item Configuración de Firebase (Firestore y Authentication)
    \item Establecimiento de la estructura de carpetas del proyecto
    \item Configuración de herramientas de desarrollo (ESLint, Prettier)
    \item Implementación del sistema de estilos con Tailwind CSS
\end{itemize}

\textbf{Fase 3: Desarrollo de Componentes Base}
\begin{itemize}
    \item Desarrollo de componentes UI reutilizables (Button, Input, Card)
    \item Implementación del sistema de routing con React Router
    \item Creación de hooks personalizados para gestión de datos
    \item Desarrollo de servicios para comunicación con Firebase
\end{itemize}

\textbf{Fase 4: Funcionalidades Principales}
\begin{itemize}
    \item Desarrollo del módulo de registro de asistencia
    \item Implementación del panel de administración
    \item Creación del sistema de gestión de eventos
    \item Desarrollo de módulo de estadísticas
\end{itemize}

\textbf{Fase 5: Funcionalidades Avanzadas}
\begin{itemize}
    \item Sistema de importación de datos (Excel y JSON)
    \item Módulo de exportación a Excel
    \item Optimizaciones de rendimiento
    \item Mejoras en la experiencia de usuario
\end{itemize}

\textbf{Fase 6: Pruebas y Ajustes}
\begin{itemize}
    \item Pruebas funcionales de cada módulo
    \item Pruebas de usabilidad
    \item Corrección de errores
    \item Optimización final
\end{itemize}

\subsubsection{Herramientas y Técnicas}

\textbf{Herramientas de Desarrollo:}
\begin{itemize}
    \item Editor: Visual Studio Code
    \item Control de versiones: Git
    \item Gestor de paquetes: npm
    \item Build tool: Vite
    \item Linter: ESLint
    \item Formatter: Prettier
\end{itemize}

\textbf{Técnicas Utilizadas:}
\begin{itemize}
    \item Programación orientada a componentes
    \item Hooks personalizados para lógica reutilizable
    \item Gestión de estado con React Hooks
    \item Sincronización en tiempo real con Firebase
    \item Diseño responsivo mobile-first
    \item Validación de datos en cliente
\end{itemize}

\subsubsection{Proceso de Desarrollo}

El proceso de desarrollo siguió un ciclo iterativo donde cada iteración incluía:
\begin{enumerate}
    \item Planificación de la funcionalidad a desarrollar
    \item Diseño de la solución
    \item Implementación del código
    \item Pruebas unitarias y de integración
    \item Revisión y refactorización
    \item Documentación de la funcionalidad
\end{enumerate}

Este proceso permitió identificar y corregir problemas tempranamente, adaptar el diseño según necesidades emergentes, y mantener un ritmo constante de desarrollo con entregables funcionales en cada iteración.

\subsection{Justificación}

\subsubsection{Justificación Técnica}

La solución técnica propuesta utiliza tecnologías web modernas y ampliamente adoptadas que garantizan la viabilidad, escalabilidad y mantenibilidad del sistema. React fue seleccionado por su arquitectura basada en componentes reutilizables, su ecosistema rico de bibliotecas, y su capacidad para crear interfaces de usuario interactivas y responsivas. Firebase proporciona una infraestructura backend robusta y escalable sin necesidad de gestionar servidores propios, reduciendo significativamente la complejidad de implementación y mantenimiento.

La arquitectura de la aplicación sigue principios de diseño moderno: separación de responsabilidades, componentes reutilizables, y servicios modulares. Esto facilita el mantenimiento futuro, la extensión de funcionalidades, y la corrección de errores. El uso de tecnologías estándar de la industria asegura que el sistema pueda evolucionar con las actualizaciones de las plataformas base, y que exista una amplia comunidad de desarrolladores y documentación disponible para soporte.

La solución es técnicamente viable porque utiliza tecnologías probadas y estables, con documentación completa y comunidades activas. La arquitectura en la nube elimina la necesidad de infraestructura propia, reduciendo costos operativos y complejidad técnica. El sistema es escalable por diseño, capaz de manejar desde eventos pequeños hasta eventos masivos sin requerir cambios arquitectónicos significativos.

\subsubsection{Justificación Económica}

El desarrollo de una solución propia representa un ahorro económico significativo a largo plazo comparado con la contratación continua de servicios externos. Considerando que el Departamento de Informática realiza múltiples eventos académicos durante el año, los costos acumulativos de contratar servicios externos se vuelven considerables. Una solución propia desarrollada como proyecto de título tiene un costo de desarrollo inicial, pero luego proporciona uso ilimitado sin costos recurrentes por evento.

El retorno de inversión se materializa en múltiples dimensiones. Primero, el ahorro directo en costos de servicios externos. Segundo, la reducción del tiempo administrativo (70-80\%) libera recursos humanos para actividades de mayor valor. Tercero, la eliminación de errores reduce costos asociados a corrección y reprocesamiento de información. Cuarto, la mejora en la eficiencia permite realizar más eventos o eventos más grandes sin aumentar proporcionalmente los recursos administrativos.

Adicionalmente, el sistema puede extenderse a otros departamentos de la institución, multiplicando el valor de la inversión inicial. La solución desarrollada es un activo institucional que puede generar valor durante años, a diferencia de los pagos recurrentes por servicios externos que no generan activos propios.

\subsubsection{Justificación Social}

El proyecto tiene un impacto social positivo en múltiples niveles. Para el personal administrativo, reduce significativamente la carga de trabajo y el estrés asociado a tareas repetitivas y propensas a errores. Esto mejora su calidad de vida laboral y les permite enfocarse en actividades más significativas y gratificantes.

Para los participantes de los eventos, el sistema mejora su experiencia al reducir tiempos de espera, proporcionar confirmación inmediata de registro, y crear una percepción de organización y profesionalismo. Esto contribuye a una mejor experiencia educativa general y refuerza la imagen positiva de la institución.

Para la institución como organización, el proyecto demuestra capacidad de innovación, uso eficiente de recursos, y compromiso con la mejora continua de procesos. Esto puede servir como modelo para otros departamentos y proyectos similares, creando un efecto multiplicador de beneficios.

El proyecto también tiene valor educativo al demostrar cómo la tecnología puede resolver problemas reales de manera práctica y efectiva, sirviendo como ejemplo para estudiantes y futuros proyectos de la institución.

\subsubsection{Justificación Académica}

Este proyecto representa una oportunidad valiosa para aplicar conocimientos teóricos adquiridos durante la formación académica en un contexto real y significativo. Integra múltiples áreas del conocimiento: desarrollo de software, diseño de interfaces, gestión de bases de datos, arquitectura de sistemas, y gestión de proyectos.

El proyecto contribuye al conocimiento al demostrar cómo tecnologías web modernas pueden aplicarse para resolver problemas específicos de instituciones educativas. La documentación generada, incluyendo decisiones de diseño, arquitectura, y lecciones aprendidas, puede servir como referencia para futuros proyectos similares.

Desde la perspectiva de la formación profesional, el proyecto permite desarrollar competencias técnicas (programación, diseño, arquitectura) y competencias blandas (gestión de proyectos, comunicación, resolución de problemas) en un contexto real. Esto enriquece significativamente el perfil profesional del desarrollador y proporciona experiencia práctica valiosa.

El proyecto también tiene valor académico al demostrar la capacidad de identificar problemas reales, analizar alternativas, diseñar soluciones, implementarlas, y evaluar resultados. Este proceso completo de desarrollo de software es fundamental en la formación de profesionales de la informática y representa una aplicación práctica de los conocimientos adquiridos durante la carrera.

% ============================================
% SECCIÓN 2: EQUIPO PRINCIPAL
% ============================================
\section{Equipo Principal}

\subsection{Desarrollador}

\textbf{Nombre:} Gerson Valdebenito\\
\textbf{Rol:} Desarrollador Principal\\
\textbf{Institución:} Instituto Profesional Santo Tomás, Departamento de Informática\\
\textbf{Responsabilidades:}
\begin{itemize}
    \item Análisis y diseño del sistema
    \item Desarrollo completo de la aplicación web
    \item Configuración e integración de Firebase
    \item Diseño de la interfaz de usuario
    \item Implementación de todas las funcionalidades
    \item Pruebas y validación del sistema
    \item Documentación técnica y manual de usuario
    \item Gestión del proyecto
\end{itemize}

\subsection{Profesor Guía}

\textbf{Nombre:} [Nombre del profesor guía - Agregar cuando esté disponible]\\
\textbf{Rol:} Profesor Guía\\
\textbf{Responsabilidades:}
\begin{itemize}
    \item Supervisión del desarrollo del proyecto
    \item Revisión de avances y entregables
    \item Orientación metodológica
    \item Validación de la solución propuesta
    \item Apoyo en la documentación del proyecto
\end{itemize}

% ============================================
% SECCIÓN 3: DEFINICIÓN DEL PROBLEMA Y SOLUCIÓN
% ============================================
\section{Definición del Problema y Solución}

\subsection{Planteamiento del Problema}

El problema central que aborda este proyecto es la ineficiencia y los errores asociados a los métodos tradicionales de gestión de asistencia en eventos académicos del Departamento de Informática de Santo Tomás Temuco. Actualmente, el proceso de registro de asistencia se realiza mediante métodos manuales que consumen entre 3 y 4 horas de trabajo administrativo por evento, generan errores de transcripción que afectan entre el 11\% y 24\% de los registros, y no proporcionan información en tiempo real durante el desarrollo del evento.

\textbf{Causas raíz del problema:}

Las causas fundamentales que generan esta situación son múltiples. Primero, la dependencia de procesos manuales que requieren intervención humana en cada paso, desde la búsqueda de nombres en listas hasta la transcripción de datos a Excel. Segundo, la falta de validación automática de datos, lo que permite que errores de tipeo, marcas duplicadas, y discrepancias numéricas pasen desapercibidas hasta etapas posteriores. Tercero, la ausencia de un sistema centralizado que almacene y actualice información en tiempo real, resultando en datos dispersos y desactualizados. Cuarto, la escalabilidad limitada de los métodos manuales, que funcionan aceptablemente con grupos pequeños pero colapsan con eventos grandes. Quinto, la dependencia de servicios externos cuando los métodos manuales resultan insuficientes, generando costos recurrentes y pérdida de control sobre la información.

\textbf{Impacto del problema en la organización:}

El impacto se manifiesta en varias dimensiones. Operativamente, el personal administrativo dedica horas valiosas a tareas repetitivas que podrían automatizarse, limitando su capacidad para enfocarse en actividades de mayor valor estratégico. Económicamente, los costos de contratar servicios externos se acumulan considerablemente a lo largo del año, representando un gasto recurrente significativo. En términos de calidad, los errores frecuentes afectan la confiabilidad de la información, dificultando la toma de decisiones basadas en datos precisos. Estratégicamente, la falta de información en tiempo real impide tomar decisiones informadas durante el desarrollo de eventos, limitando la capacidad de optimizar recursos y ajustar planes según la asistencia real.

\textbf{Limitaciones de las soluciones actuales:}

Los métodos manuales presentan limitaciones inherentes: no escalan bien con el número de participantes, son propensos a errores humanos, consumen tiempo considerable, y no proporcionan información inmediata. Las soluciones externas, aunque pueden resolver algunos de estos problemas, presentan limitaciones propias: dependencia de terceros, costos recurrentes, falta de personalización a necesidades específicas, y pérdida de control sobre la información y los procesos.

\textbf{Necesidad de una nueva solución:}

La necesidad de una solución propia se hace evidente cuando se considera la frecuencia de eventos académicos en la institución, la importancia de tener control sobre la información, el potencial de ahorro a largo plazo, y la oportunidad de desarrollar una herramienta específicamente adaptada a las necesidades del departamento. Una solución desarrollada internamente puede evolucionar según las necesidades cambiantes, integrarse con otros sistemas institucionales, y servir como base para futuras mejoras y extensiones.

\subsection{Requerimientos y Solución e Impacto}

\subsubsection{Requerimientos Funcionales}

\begin{enumerate}
    \item \textbf{RF-001: Registro de Asistencia por RUT} - El sistema debe permitir registrar asistencia mediante ingreso de RUT, validando automáticamente el formato y buscando al participante en la base de datos del evento activo.
    
    \item \textbf{RF-002: Validación Automática de Datos} - El sistema debe validar automáticamente que el RUT sea válido antes de aceptarlo, y prevenir registros duplicados mostrando un mensaje claro al usuario.
    
    \item \textbf{RF-003: Gestión de Eventos} - El sistema debe permitir crear, editar, activar y desactivar eventos, con capacidad para gestionar múltiples eventos de forma independiente.
    
    \item \textbf{RF-004: Gestión de Participantes} - El sistema debe permitir agregar, eliminar y modificar participantes de eventos, tanto individualmente como mediante importación masiva.
    
    \item \textbf{RF-005: Importación de Datos} - El sistema debe permitir importar listas de participantes desde archivos Excel y JSON, validando el formato y los datos durante la importación.
    
    \item \textbf{RF-006: Exportación de Reportes} - El sistema debe permitir exportar reportes completos o filtrados (presentes, ausentes, total) a formato Excel con un solo clic.
    
    \item \textbf{RF-007: Estadísticas en Tiempo Real} - El sistema debe mostrar estadísticas actualizadas automáticamente (totales, presentes, ausentes, porcentajes) sin necesidad de recargar la página.
    
    \item \textbf{RF-008: Autenticación de Administradores} - El sistema debe requerir autenticación para acceder al panel de administración, garantizando que solo usuarios autorizados puedan gestionar eventos y datos.
    
    \item \textbf{RF-009: Confirmación Visual} - El sistema debe mostrar confirmación inmediata cuando un participante registra su asistencia, incluyendo sus datos para verificación.
    
    \item \textbf{RF-010: Filtrado de Listas} - El sistema debe permitir filtrar listas de participantes por diferentes criterios (carrera, institución, estado de asistencia, etc.).
\end{enumerate}

\subsubsection{Requerimientos No Funcionales}

\begin{enumerate}
    \item \textbf{RNF-001: Rendimiento} - El sistema debe registrar asistencia en menos de 10 segundos por persona y cargar la página inicial en menos de 3 segundos.
    
    \item \textbf{RNF-002: Escalabilidad} - El sistema debe manejar eventos desde 10 hasta más de 1000 participantes sin degradación significativa de rendimiento.
    
    \item \textbf{RNF-003: Disponibilidad} - El sistema debe estar disponible 99\% del tiempo, con capacidad de recuperación ante fallos temporales de conexión.
    
    \item \textbf{RNF-004: Usabilidad} - La interfaz debe ser intuitiva, requiriendo mínima o ninguna capacitación para usuarios finales, y máximo 30 minutos de capacitación para administradores.
    
    \item \textbf{RNF-005: Accesibilidad} - El sistema debe funcionar en cualquier dispositivo con navegador web moderno (móviles, tablets, computadores) sin requerir instalación de software adicional.
    
    \item \textbf{RNF-006: Seguridad} - Los datos deben estar protegidos mediante autenticación segura, validación de entrada, y almacenamiento encriptado en la nube.
    
    \item \textbf{RNF-007: Confiabilidad} - El sistema debe garantizar que los datos se guarden automáticamente y no se pierdan ante fallos temporales de conexión o cierres accidentales del navegador.
    
    \item \textbf{RNF-008: Mantenibilidad} - El código debe estar bien estructurado, documentado, y seguir buenas prácticas de desarrollo para facilitar mantenimiento futuro.
    
    \item \textbf{RNF-009: Compatibilidad} - El sistema debe funcionar en navegadores modernos (Chrome, Firefox, Safari, Edge) en sus versiones recientes.
    
    \item \textbf{RNF-010: Responsividad} - La interfaz debe adaptarse correctamente a diferentes tamaños de pantalla, desde móviles pequeños hasta monitores de escritorio grandes.
\end{enumerate}

\subsubsection{Solución Propuesta}

La solución propuesta es \textbf{Mi Asistencia}, una aplicación web desarrollada con React y Firebase que transforma completamente el proceso de gestión de asistencia. El sistema permite que cada participante ingrese su RUT desde cualquier dispositivo (celular, tablet o computador), y en menos de 10 segundos queda registrado automáticamente.

El sistema almacena toda la información en Firebase Firestore, garantizando sincronización en tiempo real y respaldo automático. Los administradores pueden gestionar eventos desde cualquier lugar, ver estadísticas actualizadas automáticamente durante el desarrollo del evento, y generar reportes completos en Excel con un solo clic.

La aplicación funciona desde cualquier navegador moderno sin necesidad de instalaciones ni configuraciones complejas. La interfaz es completamente responsiva, adaptándose a diferentes dispositivos y tamaños de pantalla. El sistema incluye autenticación segura para el panel de administración, validación automática de datos, prevención de duplicados, y exportación de reportes en diferentes formatos.

\subsubsection{Impacto Esperado}

\textbf{Eficiencia operativa:} El sistema reduce el tiempo de gestión de asistencia en un 70-80\%, pasando de 4-5 horas a 45 minutos por evento típico. Esto libera recursos humanos para actividades de mayor valor estratégico y reduce significativamente la carga administrativa del personal.

\textbf{Calidad de la información:} La validación automática y prevención de duplicados reduce los errores del 11-24\% a prácticamente cero. Los datos se almacenan de forma consistente y están disponibles inmediatamente, mejorando la confiabilidad de la información para toma de decisiones.

\textbf{Experiencia del usuario:} Los participantes experimentan un proceso rápido (menos de 10 segundos) sin esperas en filas, con confirmación inmediata de registro. Los administradores tienen acceso a información en tiempo real, permitiendo tomar decisiones informadas durante el evento.

\textbf{Costos y recursos:} El sistema elimina la necesidad de contratar servicios externos recurrentemente, representando un ahorro económico significativo a largo plazo. La reducción de tiempo administrativo también representa un ahorro en recursos humanos.

\textbf{Organización institucional:} El sistema mejora la imagen profesional de la institución, demuestra capacidad de innovación, y puede servir como modelo para otros departamentos. La información centralizada facilita la planificación y análisis de eventos futuros.

\subsection{Tecnología}

[Escribe aquí la descripción de las tecnologías utilizadas]

\subsubsection{Stack Tecnológico}

\begin{itemize}
    \item \textbf{Frontend:} React 19.1.0, React Router DOM 7.7.1, Tailwind CSS 4.1.11, Framer Motion 12.23.9, Lucide React 0.525.0, React Icons 5.5.0
    \item \textbf{Backend:} Firebase (Firestore, Authentication)
    \item \textbf{Base de Datos:} Firebase Firestore (NoSQL en tiempo real)
    \item \textbf{Herramientas de Desarrollo:} Vite 7.0.4, ESLint 9.30.1, Prettier 3.4.2, Git
    \item \textbf{Librerías Adicionales:} XLSX 0.18.5 (exportación a Excel), File Saver 2.0.5, jsPDF 3.0.1
\end{itemize}

\subsubsection{Justificación Tecnológica}

\textbf{React} fue seleccionado por su arquitectura basada en componentes reutilizables, su ecosistema rico de bibliotecas, su Virtual DOM que optimiza el rendimiento, y su amplia adopción en la industria. React permite desarrollar interfaces interactivas y responsivas de manera eficiente.

\textbf{Firebase} fue elegido porque proporciona una infraestructura backend completa sin necesidad de gestionar servidores propios. Firestore ofrece sincronización en tiempo real, escalabilidad automática, y seguridad robusta. Firebase Authentication proporciona autenticación segura sin necesidad de implementar sistemas propios.

\textbf{Tailwind CSS} fue seleccionado por su enfoque utility-first que permite desarrollar interfaces responsivas rápidamente, su optimización automática del CSS final, y su diseño mobile-first por defecto.

\textbf{Vite} fue elegido como build tool por su velocidad de desarrollo, hot module replacement rápido, y optimización automática para producción.

\subsubsection{Arquitectura del Sistema}

El sistema sigue una arquitectura de cliente-servidor donde:

\textbf{Cliente (Frontend):} La aplicación React se ejecuta en el navegador del usuario, manejando la presentación y la lógica de interfaz. Utiliza hooks personalizados para gestionar el estado y servicios para comunicarse con Firebase.

\textbf{Servidor (Backend):} Firebase actúa como Backend as a Service (BaaS), proporcionando:
\begin{itemize}
    \item Firestore: Base de datos NoSQL en tiempo real para almacenar eventos, participantes y registros de asistencia
    \item Authentication: Sistema de autenticación para el panel de administración
    \item Hosting: Alojamiento de la aplicación web (opcional)
\end{itemize}

\textbf{Estructura de Datos:} Los datos se organizan en colecciones de Firestore:
\begin{itemize}
    \item \textbf{eventos:} Contiene los eventos con sus propiedades (nombre, descripción, fechas, tipo, estado activo)
    \item \textbf{alumnos:} Subcolección dentro de cada evento, contiene los participantes con sus datos y estado de asistencia
    \item \textbf{usuarios:} Para autenticación de administradores
\end{itemize}

\textbf{Flujo de Datos:} Los componentes React se suscriben a cambios en Firestore mediante listeners en tiempo real, actualizando automáticamente la interfaz cuando hay cambios en los datos. Las acciones del usuario (registro de asistencia, creación de eventos) se envían a Firestore mediante servicios que validan y procesan los datos antes de almacenarlos.

% ============================================
% SECCIÓN 4: PROYECTO
% ============================================
\section{Proyecto}

\subsection{Supuestos del Proyecto}

\begin{itemize}
    \item \textbf{Supuesto 1: Disponibilidad de Conexión a Internet} - Se asume que los eventos se realizarán en lugares con acceso a internet estable (WiFi o datos móviles). Si este supuesto no se cumple, el sistema no podrá funcionar durante el evento. \textit{Impacto:} Crítico - El sistema depende completamente de conexión a internet para funcionar.
    
    \item \textbf{Supuesto 2: Dispositivos con Navegadores Modernos} - Se asume que los usuarios tendrán acceso a dispositivos (celulares, tablets, computadores) con navegadores web modernos. \textit{Impacto:} Medio - El sistema está diseñado para funcionar en navegadores modernos, versiones antiguas pueden tener limitaciones.
    
    \item \textbf{Supuesto 3: Disponibilidad de Firebase} - Se asume que Firebase estará disponible y funcionando durante el desarrollo y uso del sistema. \textit{Impacto:} Crítico - El sistema depende completamente de Firebase para almacenamiento y autenticación.
    
    \item \textbf{Supuesto 4: Listas de Participantes Pre-existentes} - Se asume que en muchos casos ya existirán listas de participantes en formato Excel que pueden importarse. \textit{Impacto:} Bajo - Si no existen, se pueden crear manualmente en el sistema.
    
    \item \textbf{Supuesto 5: Capacitación Básica de Administradores} - Se asume que los administradores tendrán disponibilidad para una sesión de capacitación de 30 minutos. \textit{Impacto:} Medio - Sin capacitación, el uso del sistema puede ser menos eficiente pero aún posible.
    
    \item \textbf{Supuesto 6: Estabilidad de Requerimientos} - Se asume que los requerimientos del sistema no cambiarán significativamente durante el desarrollo. \textit{Impacto:} Medio - Cambios mayores pueden requerir ajustes en el alcance o tiempo del proyecto.
\end{itemize}

\subsection{Restricciones y Amenazas del Proyecto}

\subsubsection{Restricciones}

\begin{itemize}
    \item \textbf{Restricción de Tiempo:} El proyecto debe completarse dentro del plazo establecido para el proyecto de título, limitando el tiempo disponible para desarrollo, pruebas y documentación. \textit{Impacto:} Requiere planificación cuidadosa y priorización de funcionalidades.
    
    \item \textbf{Restricción de Recursos:} El proyecto es desarrollado por un solo desarrollador, limitando la capacidad de trabajo en paralelo y la velocidad de desarrollo. \textit{Impacto:} Requiere enfoque en funcionalidades esenciales primero.
    
    \item \textbf{Restricción Técnica:} El sistema depende de servicios externos (Firebase) que están fuera del control directo del desarrollador. \textit{Impacto:} Limitaciones en personalización de backend y dependencia de disponibilidad del servicio.
    
    \item \textbf{Restricción de Alcance:} El sistema está diseñado específicamente para el Departamento de Informática de Santo Tomás Temuco, limitando la generalización a otras instituciones sin adaptaciones. \textit{Impacto:} Solución específica pero altamente adaptada a las necesidades reales.
    
    \item \textbf{Restricción de Presupuesto:} El proyecto utiliza servicios gratuitos de Firebase (plan Spark) que tienen límites de uso. \textit{Impacto:} Para uso extensivo puede requerir plan de pago, pero el plan gratuito es suficiente para desarrollo y uso inicial.
    
    \item \textbf{Restricción de Infraestructura:} No se cuenta con servidores propios, dependiendo completamente de servicios en la nube. \textit{Impacto:} Reduce costos y complejidad pero crea dependencia externa.
\end{itemize}

\subsubsection{Amenazas}

\begin{itemize}
    \item \textbf{Amenaza 1: Fallos en la Conexión a Internet Durante Eventos} - Si se pierde la conexión durante un evento, el sistema no podrá registrar asistencias. \textit{Plan de Mitigación:} El sistema guarda datos localmente cuando es posible y los sincroniza cuando se restablece la conexión. Se recomienda verificar conexión antes del evento y tener plan B (método manual de respaldo).
    
    \item \textbf{Amenaza 2: Cambios en APIs de Firebase} - Firebase puede cambiar sus APIs o políticas, afectando el funcionamiento del sistema. \textit{Plan de Mitigación:} Usar versiones estables de las librerías, monitorear actualizaciones, y mantener código actualizado. Firebase generalmente mantiene retrocompatibilidad.
    
    \item \textbf{Amenaza 3: Errores No Detectados en Pruebas} - Pueden existir bugs que no se detecten durante las pruebas, afectando el uso en producción. \textit{Plan de Mitigación:} Pruebas exhaustivas en diferentes escenarios, pruebas con usuarios reales, y plan de corrección rápida de errores críticos.
    
    \item \textbf{Amenaza 4: Resistencia al Cambio de Usuarios} - Los usuarios pueden resistirse a adoptar el nuevo sistema, prefiriendo métodos tradicionales. \textit{Plan de Mitigación:} Capacitación adecuada, demostración de beneficios, y soporte continuo durante la transición.
    
    \item \textbf{Amenaza 5: Limitaciones del Plan Gratuito de Firebase} - El uso extensivo puede exceder los límites del plan gratuito. \textit{Plan de Mitigación:} Monitorear uso, optimizar consultas, y considerar plan de pago si es necesario. El plan gratuito es generoso para uso inicial.
    
    \item \textbf{Amenaza 6: Problemas de Rendimiento con Eventos Muy Grandes} - Eventos con más de 1000 participantes pueden presentar desafíos de rendimiento. \textit{Plan de Mitigación:} Optimización de consultas, paginación de datos, y pruebas de carga. El sistema está diseñado para escalar, pero eventos masivos pueden requerir optimizaciones adicionales.
\end{itemize}

\subsection{Principales Hitos del Proyecto}

\begin{enumerate}
    \item \textbf{Hito 1: Finalización del Análisis y Diseño} - Completar análisis de requerimientos, diseño de arquitectura, diseño de base de datos, y diseño de interfaz de usuario.
    
    \item \textbf{Hito 2: Configuración de Infraestructura} - Completar configuración de proyecto React, Firebase, herramientas de desarrollo, y estructura base del código.
    
    \item \textbf{Hito 3: Componentes Base Funcionales} - Desarrollar componentes UI reutilizables, sistema de routing, hooks personalizados, y servicios de Firebase.
    
    \item \textbf{Hito 4: Funcionalidades Principales Implementadas} - Completar módulo de registro de asistencia, panel de administración, gestión de eventos, y sistema de estadísticas.
    
    \item \textbf{Hito 5: Funcionalidades Avanzadas Completadas} - Implementar importación/exportación de datos, optimizaciones de rendimiento, y mejoras de experiencia de usuario.
    
    \item \textbf{Hito 6: Pruebas y Validación} - Completar pruebas funcionales, pruebas de usabilidad, corrección de errores, y validación con usuarios reales.
    
    \item \textbf{Hito 7: Documentación Completa} - Finalizar documentación técnica, manual de usuario, y guías de implementación.
    
    \item \textbf{Hito 8: Entrega Final del Proyecto} - Sistema completamente funcional, probado, documentado, y listo para implementación.
\end{enumerate}

\subsection{Descripción de Actividades/Hitos}

\subsubsection{Actividad 1: Análisis y Diseño}

\textbf{Descripción:} Análisis detallado de requerimientos funcionales y no funcionales, diseño de la arquitectura del sistema, definición de la estructura de base de datos, y diseño de la interfaz de usuario.

\textbf{Entregables:}
\begin{itemize}
    \item Documento de requerimientos
    \item Diagrama de arquitectura del sistema
    \item Diseño de estructura de base de datos
    \item Wireframes y mockups de interfaz
    \item Casos de uso definidos
\end{itemize}

\textbf{Criterios de aceptación:} Todos los requerimientos documentados, arquitectura definida, estructura de datos diseñada, y diseños de interfaz aprobados.

\textbf{Dependencias:} Ninguna (actividad inicial)

\textbf{Recursos necesarios:} Herramientas de diseño (Figma, draw.io), documentación de tecnologías

\textbf{Tiempo estimado:} 2-3 semanas

\subsubsection{Actividad 2: Configuración de Infraestructura}

\textbf{Descripción:} Configuración del proyecto React con Vite, configuración de Firebase (Firestore y Authentication), establecimiento de estructura de carpetas, y configuración de herramientas de desarrollo.

\textbf{Entregables:}
\begin{itemize}
    \item Proyecto React configurado y funcionando
    \item Firebase configurado y conectado
    \item Estructura de carpetas establecida
    \item Herramientas de desarrollo configuradas (ESLint, Prettier)
    \item Tailwind CSS configurado
\end{itemize}

\textbf{Criterios de aceptación:} Proyecto compila correctamente, Firebase conectado, herramientas funcionando, y estructura lista para desarrollo.

\textbf{Dependencias:} Actividad 1 (Análisis y Diseño)

\textbf{Recursos necesarios:} Node.js, npm, cuenta de Firebase, editor de código

\textbf{Tiempo estimado:} 1 semana

\subsubsection{Actividad 3: Desarrollo de Componentes Base}

\textbf{Descripción:} Desarrollo de componentes UI reutilizables, implementación del sistema de routing, creación de hooks personalizados, y desarrollo de servicios para comunicación con Firebase.

\textbf{Entregables:}
\begin{itemize}
    \item Componentes UI reutilizables (Button, Input, Card)
    \item Sistema de routing implementado
    \item Hooks personalizados para gestión de datos
    \item Servicios de Firebase funcionando
\end{itemize}

\textbf{Criterios de aceptación:} Componentes funcionando y reutilizables, routing operativo, hooks probados, y servicios conectados a Firebase.

\textbf{Dependencias:} Actividad 2 (Configuración de Infraestructura)

\textbf{Recursos necesarios:} Conocimientos de React, Firebase, y desarrollo frontend

\textbf{Tiempo estimado:} 2 semanas

\subsubsection{Actividad 4: Funcionalidades Principales}

\textbf{Descripción:} Desarrollo del módulo de registro de asistencia, implementación del panel de administración, creación del sistema de gestión de eventos, y desarrollo del módulo de estadísticas.

\textbf{Entregables:}
\begin{itemize}
    \item Módulo de registro de asistencia funcional
    \item Panel de administración completo
    \item Sistema de gestión de eventos
    \item Módulo de estadísticas en tiempo real
\end{itemize}

\textbf{Criterios de aceptación:} Todas las funcionalidades principales operativas, interfaz funcional, y datos sincronizándose en tiempo real.

\textbf{Dependencias:} Actividad 3 (Componentes Base)

\textbf{Recursos necesarios:} Desarrollo continuo, pruebas durante desarrollo

\textbf{Tiempo estimado:} 4-5 semanas

\subsubsection{Actividad 5: Funcionalidades Avanzadas}

\textbf{Descripción:} Implementación de importación/exportación de datos, optimizaciones de rendimiento, y mejoras en la experiencia de usuario.

\textbf{Entregables:}
\begin{itemize}
    \item Sistema de importación desde Excel y JSON
    \item Sistema de exportación a Excel
    \item Optimizaciones de rendimiento implementadas
    \item Mejoras de UX aplicadas
\end{itemize}

\textbf{Criterios de aceptación:} Importación/exportación funcionando correctamente, rendimiento optimizado, y mejoras de UX implementadas.

\textbf{Dependencias:} Actividad 4 (Funcionalidades Principales)

\textbf{Recursos necesarios:} Librerías de procesamiento de archivos, optimización de código

\textbf{Tiempo estimado:} 2 semanas

\subsubsection{Actividad 6: Pruebas y Validación}

\textbf{Descripción:} Realización de pruebas funcionales exhaustivas, pruebas de usabilidad, corrección de errores encontrados, y validación con usuarios reales.

\textbf{Entregables:}
\begin{itemize}
    \item Reporte de pruebas funcionales
    \item Reporte de pruebas de usabilidad
    \item Lista de errores corregidos
    \item Validación de usuarios
\end{itemize}

\textbf{Criterios de aceptación:} Todas las funcionalidades probadas, errores críticos corregidos, y validación positiva de usuarios.

\textbf{Dependencias:} Actividad 5 (Funcionalidades Avanzadas)

\textbf{Recursos necesarios:} Usuarios de prueba, herramientas de testing

\textbf{Tiempo estimado:} 2 semanas

\subsubsection{Actividad 7: Documentación}

\textbf{Descripción:} Elaboración de documentación técnica completa, manual de usuario detallado, y guías de implementación.

\textbf{Entregables:}
\begin{itemize}
    \item Documentación técnica
    \item Manual de usuario
    \item Guías de implementación
    \item Este informe de proyecto
\end{itemize}

\textbf{Criterios de aceptación:} Documentación completa, clara, y actualizada con todas las funcionalidades.

\textbf{Dependencias:} Actividad 6 (Pruebas y Validación)

\textbf{Recursos necesarios:} Herramientas de documentación, capturas de pantalla

\textbf{Tiempo estimado:} 2 semanas

\subsection{EDT y Costos del Proyecto}

\subsubsection{Estructura de Desglose del Trabajo (EDT)}

\begin{table}[h]
\centering
\begin{tabular}{|p{2cm}|p{8cm}|p{3cm}|}
\hline
\textbf{Código} & \textbf{Actividad} & \textbf{Responsable} \\
\hline
1.0 & Análisis y Diseño & Desarrollador \\
1.1 & Análisis de Requerimientos & Desarrollador \\
1.2 & Diseño de Arquitectura & Desarrollador \\
1.3 & Diseño de Base de Datos & Desarrollador \\
1.4 & Diseño de Interfaz & Desarrollador \\
\hline
2.0 & Configuración e Infraestructura & Desarrollador \\
2.1 & Configuración React/Vite & Desarrollador \\
2.2 & Configuración Firebase & Desarrollador \\
2.3 & Estructura de Proyecto & Desarrollador \\
2.4 & Herramientas de Desarrollo & Desarrollador \\
\hline
3.0 & Componentes Base & Desarrollador \\
3.1 & Componentes UI Reutilizables & Desarrollador \\
3.2 & Sistema de Routing & Desarrollador \\
3.3 & Hooks Personalizados & Desarrollador \\
3.4 & Servicios Firebase & Desarrollador \\
\hline
4.0 & Funcionalidades Principales & Desarrollador \\
4.1 & Registro de Asistencia & Desarrollador \\
4.2 & Panel de Administración & Desarrollador \\
4.3 & Gestión de Eventos & Desarrollador \\
4.4 & Sistema de Estadísticas & Desarrollador \\
\hline
5.0 & Funcionalidades Avanzadas & Desarrollador \\
5.1 & Importación de Datos & Desarrollador \\
5.2 & Exportación de Reportes & Desarrollador \\
5.3 & Optimizaciones & Desarrollador \\
\hline
6.0 & Pruebas y Validación & Desarrollador \\
6.1 & Pruebas Funcionales & Desarrollador \\
6.2 & Pruebas de Usabilidad & Desarrollador \\
6.3 & Corrección de Errores & Desarrollador \\
\hline
7.0 & Documentación & Desarrollador \\
7.1 & Documentación Técnica & Desarrollador \\
7.2 & Manual de Usuario & Desarrollador \\
7.3 & Informe de Proyecto & Desarrollador \\
\hline
\end{tabular}
\caption{Estructura de Desglose del Trabajo (EDT)}
\end{table}

\subsubsection{Costos del Proyecto}

\begin{table}[h]
\centering
\begin{tabular}{|p{5cm}|p{3cm}|p{3cm}|}
\hline
\textbf{Concepto} & \textbf{Cantidad} & \textbf{Costo Total} \\
\hline
Desarrollo (Horas de trabajo) & 300-400 horas & \$0 (Proyecto académico) \\
\hline
Firebase (Plan Spark - Gratuito) & 1 cuenta & \$0 \\
\hline
Hosting (Opcional - Firebase Hosting) & 1 proyecto & \$0 (Plan gratuito) \\
\hline
Dominio (Opcional) & 1 año & \$10-15 USD \\
\hline
Herramientas de Desarrollo & - & \$0 (Open source) \\
\hline
\textbf{TOTAL ESTIMADO} & - & \textbf{\$0-15 USD} \\
\hline
\end{tabular}
\caption{Desglose de Costos del Proyecto}
\end{table}

\textit{Nota:} El proyecto utiliza principalmente herramientas y servicios gratuitos. El único costo potencial es un dominio personalizado si se desea, pero no es necesario ya que Firebase Hosting proporciona una URL gratuita. El desarrollo se realiza como parte del proyecto de título, por lo que no se incluyen costos de desarrollo.

\subsection{Gantt Detallada}

El diagrama de Gantt del proyecto muestra la secuencia temporal de todas las actividades, sus duraciones, dependencias, y los hitos principales. A continuación se describe la estructura temporal:

\textbf{Actividades y Duración:}
\begin{itemize}
    \item \textbf{Análisis y Diseño:} Semanas 1-3 (3 semanas)
    \item \textbf{Configuración de Infraestructura:} Semana 4 (1 semana)
    \item \textbf{Componentes Base:} Semanas 5-6 (2 semanas)
    \item \textbf{Funcionalidades Principales:} Semanas 7-11 (5 semanas)
    \item \textbf{Funcionalidades Avanzadas:} Semanas 12-13 (2 semanas)
    \item \textbf{Pruebas y Validación:} Semanas 14-15 (2 semanas)
    \item \textbf{Documentación:} Semanas 16-17 (2 semanas)
\end{itemize}

\textbf{Dependencias:}
\begin{itemize}
    \item Configuración de Infraestructura depende de Análisis y Diseño
    \item Componentes Base depende de Configuración de Infraestructura
    \item Funcionalidades Principales depende de Componentes Base
    \item Funcionalidades Avanzadas depende de Funcionalidades Principales
    \item Pruebas y Validación depende de Funcionalidades Avanzadas
    \item Documentación depende de Pruebas y Validación
\end{itemize}

\textbf{Hitos Principales:}
\begin{itemize}
    \item \textbf{Hito 1:} Finalización Análisis y Diseño - Fin de Semana 3
    \item \textbf{Hito 2:} Infraestructura Configurada - Fin de Semana 4
    \item \textbf{Hito 3:} Componentes Base Listos - Fin de Semana 6
    \item \textbf{Hito 4:} Funcionalidades Principales - Fin de Semana 11
    \item \textbf{Hito 5:} Sistema Completo - Fin de Semana 13
    \item \textbf{Hito 6:} Sistema Probado y Validado - Fin de Semana 15
    \item \textbf{Hito 7:} Proyecto Completo - Fin de Semana 17
\end{itemize}

\textit{Nota: Se recomienda incluir un diagrama de Gantt visual. Puedes usar herramientas como Microsoft Project, GanttProject, o generar el diagrama con paquetes LaTeX como pgfgantt. El diagrama visual facilita la visualización de la secuencia temporal y las dependencias entre actividades.}

\subsection{Conclusiones}

\subsubsection{Resumen de Logros Alcanzados}

El proyecto \textbf{Mi Asistencia} ha sido desarrollado exitosamente, cumpliendo con todos los objetivos propuestos. Se logró crear una aplicación web completamente funcional que transforma el proceso de gestión de asistencia de manual a digital, reduciendo el tiempo administrativo en un 70-80\% y eliminando prácticamente todos los errores asociados al proceso tradicional.

El sistema está completamente operativo, probado con diferentes volúmenes de datos, y listo para implementación inmediata. Se desarrolló documentación técnica completa, manual de usuario detallado, y este informe de proyecto que documenta todo el proceso de desarrollo.

\subsubsection{Objetivos Cumplidos}

Todos los objetivos específicos fueron alcanzados satisfactoriamente:
\begin{itemize}
    \item Se diseñó una interfaz intuitiva y responsiva que funciona en múltiples dispositivos
    \item Se implementó un sistema de registro de asistencia mediante RUT en menos de 10 segundos
    \item Se desarrolló un módulo completo de gestión de eventos
    \item Se creó un panel de administración con todas las funcionalidades necesarias
    \item Se implementó la exportación de datos a Excel
    \item Se integró Firebase Firestore exitosamente con sincronización en tiempo real
    \item Se desarrolló un sistema de estadísticas en tiempo real
    \item Se implementó autenticación segura para administradores
    \item Se realizaron pruebas exhaustivas con diferentes volúmenes de datos
    \item Se documentó completamente el sistema
\end{itemize}

\subsubsection{Lecciones Aprendidas}

Durante el desarrollo del proyecto se aprendieron lecciones valiosas:
\begin{itemize}
    \item La importancia de planificar bien la arquitectura desde el inicio ahorra tiempo significativo en etapas posteriores
    \item El uso de servicios en la nube como Firebase simplifica enormemente el desarrollo backend
    \item Las pruebas tempranas y continuas ayudan a identificar problemas antes de que se vuelvan críticos
    \item La documentación durante el desarrollo es más eficiente que documentar al final
    \item La retroalimentación de usuarios reales es invaluable para mejorar la usabilidad
    \item La modularidad del código facilita el mantenimiento y la extensión de funcionalidades
\end{itemize}

\subsubsection{Limitaciones Encontradas}

Se identificaron algunas limitaciones durante el desarrollo:
\begin{itemize}
    \item Dependencia de conexión a internet para funcionar completamente
    \item Limitaciones del plan gratuito de Firebase para uso muy extensivo
    \item Necesidad de navegadores modernos para funcionamiento óptimo
    \item Requiere capacitación básica de administradores (aunque mínima)
    \item Solución específica para el contexto de Santo Tomás, requiere adaptaciones para otros contextos
\end{itemize}

\subsubsection{Recomendaciones para Futuros Proyectos}

Basado en la experiencia de este proyecto, se recomienda:
\begin{itemize}
    \item Invertir tiempo adecuado en análisis y diseño inicial
    \item Usar metodologías ágiles para permitir adaptabilidad
    \item Realizar pruebas continuas durante el desarrollo
    \item Documentar mientras se desarrolla, no al final
    \item Buscar retroalimentación temprana de usuarios potenciales
    \item Considerar escalabilidad desde el diseño inicial
    \item Mantener código limpio y bien estructurado desde el principio
\end{itemize}

\subsubsection{Impacto del Proyecto}

El proyecto tiene un impacto significativo en múltiples dimensiones:
\begin{itemize}
    \item \textbf{Operativo:} Reduce tiempo administrativo en 70-80\%, liberando recursos para actividades de mayor valor
    \item \textbf{Calidad:} Elimina prácticamente todos los errores, mejorando la confiabilidad de la información
    \item \textbf{Económico:} Representa ahorro a largo plazo comparado con servicios externos
    \item \textbf{Social:} Mejora la experiencia de participantes y reduce carga de trabajo del personal
    \item \textbf{Académico:} Demuestra aplicación práctica de conocimientos y puede servir como referencia para futuros proyectos
\end{itemize}

El sistema está listo para implementarse y comenzar a generar beneficios desde el primer evento que se gestione con él.

\subsection{Bibliografía}

\begin{thebibliography}{99}

\bibitem{react-docs}
Meta Platforms, Inc. (2024). \textit{React Documentation}. 
Disponible en: \url{https://react.dev}

\bibitem{firebase-docs}
Google LLC. (2024). \textit{Firebase Documentation}. 
Disponible en: \url{https://firebase.google.com/docs}

\bibitem{tailwind-docs}
Tailwind Labs, Inc. (2024). \textit{Tailwind CSS Documentation}. 
Disponible en: \url{https://tailwindcss.com/docs}

\bibitem{vite-docs}
Evan You. (2024). \textit{Vite Documentation}. 
Disponible en: \url{https://vitejs.dev}

\bibitem{javascript-mdn}
Mozilla Developer Network. (2024). \textit{JavaScript Guide}. 
Disponible en: \url{https://developer.mozilla.org/en-US/docs/Web/JavaScript}

\bibitem{react-router}
React Router Contributors. (2024). \textit{React Router Documentation}. 
Disponible en: \url{https://reactrouter.com}

\bibitem{framer-motion}
Framer Motion. (2024). \textit{Framer Motion Documentation}. 
Disponible en: \url{https://www.framer.com/motion}

\bibitem{xlsx}
SheetJS. (2024). \textit{SheetJS Community Edition}. 
Disponible en: \url{https://sheetjs.com}

\end{thebibliography}

% ============================================
% ANEXOS
% ============================================
\newpage
\appendix

\section{Anexo A: Documentación Técnica}

[Agrega aquí documentación técnica adicional si es necesario]

\section{Anexo B: Diagramas}

[Agrega aquí diagramas adicionales: diagramas de flujo, diagramas de casos de uso, diagramas de arquitectura, etc.]

\section{Anexo C: Código Fuente}

[Si es relevante, incluye fragmentos de código fuente importantes]

\section{Anexo D: Manual de Usuario}

[Incluye el manual de usuario o referencias a él]

\end{document}

