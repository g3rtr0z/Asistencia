\documentclass[12pt,a4paper]{article}

% Paquetes necesarios
\usepackage[utf8]{inputenc}
\usepackage[spanish]{babel}
\usepackage{graphicx}
\usepackage{hyperref}
\usepackage{xcolor}
\usepackage{geometry}
\usepackage{fancyhdr}
\usepackage{titlesec}
\usepackage{enumitem}
\usepackage{tikz}
\usepackage{fontawesome5}

% Configuración de página
\geometry{
    a4paper,
    left=2.5cm,
    right=2.5cm,
    top=3cm,
    bottom=2.5cm
}

% Configuración de encabezados y pies de página
\pagestyle{fancy}
\fancyhf{}
\fancyhead[L]{\leftmark}
\fancyhead[R]{\thepage}
\fancyfoot[C]{Mi Asistencia - Sistema de Gestión de Asistencia}

% Colores corporativos
\definecolor{verdest}{RGB}{0, 99, 65}
\definecolor{verdeclarost}{RGB}{0, 179, 136}
\definecolor{grisclaro}{RGB}{245, 245, 245}

% Títulos personalizados
\titleformat{\section}
{\Large\bfseries\color{verdest}}
{\thesection}{1em}{}

\titleformat{\subsection}
{\large\bfseries\color{verdeclarost}}
{\thesubsection}{1em}{}

% Información del documento
\title{Presentación del Sistema\\\large Mi Asistencia\\\large Gestión Eficiente de Asistencia para Eventos}
\author{Departamento de Informática\\Santo Tomás Temuco}
\date{\today}

\begin{document}

% Portada
\begin{titlepage}
    \centering
    \vspace*{0.5cm}
    
    \includegraphics[width=0.25\textwidth]{src/assets/logopag.png}\\[1cm]
    
    {\Huge\bfseries\color{verdest} Sistema de Gestión de\\Asistencia para Eventos}\\[0.5cm]
    {\LARGE\itshape Mi Asistencia}\\[1.5cm]
    
    {\large Solución Integral para la Gestión Eficiente de\\Participantes en Eventos Académicos}\\[2cm]
    
    {\large Presentado a:}\\[0.5cm]
    {\Large Docentes del Departamento de Informática}\\[1cm]
    
    {\large Departamento de Informática}\\[0.5cm]
    {\large Instituto Profesional Santo Tomás}\\[0.5cm]
    {\large Sede Temuco}\\[2cm]
    
    {\large\date\today}
    
    \vfill
    
    {\small Desarrollado por: Gerson Valdebenito}
\end{titlepage}

% Resumen Ejecutivo
\newpage
\section*{Resumen Ejecutivo}
\addcontentsline{toc}{section}{Resumen Ejecutivo}

\textbf{Mi Asistencia} es una aplicación web que desarrollé pensando en las necesidades reales del Departamento de Informática de Santo Tomás Temuco. La idea surgió después de observar los problemas que enfrentábamos al gestionar la asistencia en los eventos del departamento.

En lugar de seguir usando listas en papel o hojas de cálculo que todos sabemos que pueden ser un dolor de cabeza, creé esta plataforma para que el registro de asistencia sea algo rápido y simple. Lo mejor es que funciona desde cualquier dispositivo con internet, así que no necesitas estar en un lugar específico para usarla.

\vspace{0.5cm}

Cuando empecé a trabajar en esto, pensé en qué es lo que realmente necesitábamos. No quería hacer algo complicado solo para impresionar, sino algo que realmente funcionara en la práctica. Por eso me enfoqué en que ahorrara tiempo (mucho tiempo, la verdad), que diera información al instante sin tener que esperar, que evitara esos errores tontos que todos cometemos cuando escribimos manualmente, y que funcionara en lo que sea que tenga la persona a mano, ya sea su celular, una tablet o un computador. Además, que al final puedas sacar los reportes rápido sin tener que pasarte horas copiando datos de un lado a otro.

% Índice
\newpage
\tableofcontents
\newpage

% Introducción
\section{Introducción}

\subsection{El Problema que Resolvemos}

Creo que todos hemos pasado por esto alguna vez: estás en un evento con mucha gente y tienes que ir marcando manualmente quién llegó y quién no. Es tedioso, toma mucho tiempo y además es fácil equivocarse.

Los métodos tradicionales que usamos tienen varios problemas que seguro todos conocemos:

\begin{itemize}
    \item \textbf{Se pierde mucho tiempo:} Tienes que estar ahí marcando uno por uno, y si son muchos participantes, te puede tomar horas
    \item \textbf{Errores comunes:} A veces escribes mal un RUT, o marcas dos veces a la misma persona, o simplemente se te olvida quién ya llegó
    \item \textbf{No sabes nada hasta el final:} Mientras pasa el evento, no tienes idea de cuántos han llegado realmente. Solo lo sabes cuando terminas de contar manualmente
    \item \textbf{Hacer reportes es complicado:} Después tienes que sentarte a pasar todo a Excel, lo que significa más tiempo y más posibilidades de error
    \item \textbf{Si pierdes la hoja, perdiste todo:} Las listas en papel se pueden perder, mojar, o simplemente quedar ilegibles
\end{itemize}

\subsection{La Solución: Mi Asistencia}

Por eso decidí crear \textbf{Mi Asistencia}. Es una aplicación web sencilla pero potente que resuelve todos esos problemas que mencioné.

La idea es simple: en lugar de estar con papel y lápiz, cada persona ingresa su RUT en un celular, tablet o computador, y listo. El sistema registra todo automáticamente y al instante puedes ver cuánta gente ha llegado. Los administradores pueden gestionar los eventos desde cualquier lugar, y lo mejor es que toda la información está guardada en la nube, así que no se pierde nada.

\subsection{Objetivos de la Solución}

Cuando empecé este proyecto, me puse algunos objetivos claros. El primero era hacer que todo fuera simple. Quería que registrar asistencia fuera tan fácil como escribir un RUT y listo, nada más. También quería que la información estuviera disponible al momento, no tener que esperar hasta el final del evento para saber cuántos vinieron.

Otro objetivo importante era ahorrar tiempo. Me di cuenta que pasábamos demasiadas horas haciendo cosas que perfectamente podrían automatizarse. Y claro, reducir los errores también era clave. Por último, pensé en que fuera fácil analizar los datos después, por eso incluí la exportación a Excel que es el formato que todos usamos.

% Características Principales
\section{Características Principales}

\subsection{Interfaz Intuitiva y Moderna}

Lo importante para mí era que fuera fácil de usar. Quería que cualquier persona, incluso si no está acostumbrada a usar tecnología, pudiera entenderlo rápidamente. Por eso diseñé la interfaz para que sea clara y directa.

Seguí los colores de Santo Tomás para que se sienta familiar cuando lo vean, pero mantuve todo limpio y sin distracciones innecesarias. Intenté que todo esté donde esperarías que esté, así no tienes que adivinar dónde hacer clic o buscar funciones ocultas en menús complicados. Probé que funcionara bien en celulares, tablets y computadores porque nunca sabes qué dispositivo va a usar la persona. Y cuando alguien marca su asistencia, ve inmediatamente que quedó registrado, así no hay dudas ni tienen que preguntar si funcionó o no.

\subsection{Registro Rápido y Eficiente}

El proceso es realmente simple. La persona ingresa su RUT (puede ser con o sin puntos y guión, el sistema lo entiende de cualquier forma). El sistema busca automáticamente en la base de datos del evento, y si está registrado, marca la asistencia al instante. Aparece una pantalla de confirmación mostrando sus datos para que sepan que quedó registrado.

Todo esto tarda menos de 10 segundos. Literalmente puedes registrar a 6 personas en un minuto, cosa que con el método tradicional sería imposible. Es como la diferencia entre escribir a mano y usar un procesador de texto.

\subsection{Información en Tiempo Real}

Una de las cosas que más me gusta del sistema es que puedes ver todo en tiempo real. Esto es súper útil porque durante el evento puedes ir viendo cómo va la asistencia sin tener que esperar.

Ves de un vistazo cuántos participantes están registrados en total. Puedes ver cuántos ya llegaron en ese momento exacto. Y también ves quiénes faltan, así sabes si esperar más gente o si ya puedes empezar. Los números se van actualizando solos, no tienes que estar recargando la página constantemente como pasa con otras aplicaciones. Simplemente dejas la pantalla abierta y ves cómo van cambiando los números a medida que más gente va llegando.

\subsection{Gestión Completa de Eventos}

Puedes tener múltiples eventos en el sistema sin problema. Cada uno es independiente, así que puedes planificar varios eventos con anticipación y simplemente activar el que necesites en el momento.

\begin{itemize}
    \item \textbf{Crear eventos es fácil:} Solo necesitas ponerle un nombre, las fechas, una descripción breve y decir si es para alumnos o funcionarios
    \item \textbf{Un evento a la vez:} Solo puede haber un evento activo para evitar confusiones. Cuando termina uno, activas el siguiente
    \item \textbf{Gestionar participantes:} Puedes agregar personas de a una, o si tienes una lista grande, la importas completa desde Excel
    \item \textbf{Dos tipos de eventos:} El sistema distingue entre eventos para alumnos y para funcionarios, porque a veces necesitas información diferente
\end{itemize}

\subsection{Importación y Exportación de Datos}

Cuando tienes que agregar muchos participantes de una vez, esto puede ser un problema. Por eso le agregué la opción de importar desde Excel. Muchas veces ya tienen las listas en Excel de antes, así que simplemente cargas el archivo y el sistema lo procesa. También acepta JSON por si alguien ya tiene sus datos en ese formato, aunque honestamente casi nadie lo usa.

Para exportar, funciona similar. Puedes sacar todo el reporte completo o si solo necesitas ver quiénes llegaron o quiénes faltaron, puedes filtrar antes de exportar. Todo sale en Excel porque es el formato que todos sabemos usar.

\subsection{Panel de Administración Completo}

El panel de administración es donde los administradores pueden hacer todo. Es como el centro de control. Desde ahí gestionan los eventos, ven quiénes están registrados, pueden agregar o quitar personas si falta alguien o sobra alguien, y lo mejor es que todo el tiempo pueden estar viendo en tiempo real cómo va la asistencia. Cuando necesitan un reporte, simplemente lo descargan. No tienen que andar buscando en varios lugares, todo está ahí.

% Beneficios y Ventajas
\section{Beneficios y Ventajas}

\subsection{Beneficios Operativos}

\subsubsection{Ahorro de Tiempo}

Este es quizás el beneficio más obvio, pero también el más importante. Cuando empecé a desarrollar esto, pensé en cuánto tiempo perdíamos haciendo cosas que podrían ser automáticas:

\begin{itemize}
    \item \textbf{Registro súper rápido:} Lo que antes te tomaba minutos, ahora toma literalmente segundos. Puedes registrar a alguien en menos tiempo del que se tarda en encontrar su nombre en una lista
    \item \textbf{No más transcribir:} Ya no tienes que pasar horas después del evento transcribiendo todo a Excel. El sistema ya tiene todo guardado
    \item \textbf{Reportes automáticos:} Cuando necesites un reporte, lo descargas con un clic. No más copiar y pegar datos
    \item \textbf{Múltiples personas pueden ayudar:} Si hay muchas personas llegando a la vez, pueden abrir el sistema en varios celulares y todos registrar al mismo tiempo. Es genial
\end{itemize}

\subsubsection{Reducción de Errores}

El sistema valida automáticamente que el RUT sea correcto antes de aceptarlo. Si alguien intenta marcar dos veces la misma persona, el sistema no lo deja, te avisa que esa persona ya está registrada. Toda la información se guarda de la misma forma siempre, así que no hay inconsistencias. Y como los datos se toman directamente de la base de datos, no hay errores de transcripción, que esos siempre aparecen cuando pasas cosas de un lado a otro manualmente.

\subsubsection{Accesibilidad}

Esta parte es importante porque no todos tienen el mismo tipo de dispositivo. Algunos prefieren usar su celular, otros una tablet, otros el computador. Por eso hice que funcionara bien en todos. No importa qué tengas, si tiene navegador web funciona.

Y no hay que instalar nada, que eso siempre es complicado. Solo abres la página en tu navegador, ya sea Chrome, Firefox, Safari, el que uses. Mientras tengas internet, funciona. Lo bueno es que varias personas pueden estar usando el sistema al mismo tiempo desde diferentes lugares, todos registrando asistencia en paralelo. Eso hace que sea mucho más rápido cuando hay mucha gente llegando a la vez.

\subsection{Beneficios Estratégicos}

\subsubsection{Toma de Decisiones Informada}

Esto es algo que no me había dado cuenta que sería tan útil hasta que lo probé. Tener la información en tiempo real te permite reaccionar mientras el evento está pasando:

\begin{itemize}
    \item Ves al instante cuántos han llegado. Si faltan 10 minutos para empezar y ves que han llegado pocos, puedes avisar por WhatsApp o hacer recordatorios
    \item Sabes si las expectativas se están cumpliendo. Si esperabas 100 personas y a la hora de inicio solo han llegado 30, puedes ajustar tu plan
    \item Puedes tomar decisiones sobre la marcha. Por ejemplo, si ves que la asistencia es baja, puedes extender el tiempo de registro o ajustar el programa
    \item Para eventos futuros, tienes datos reales. Ya no tienes que adivinar cuánta gente va a venir basándote en nada, tienes historial real
\end{itemize}

\subsubsection{Mejora en la Organización}

Cuando planificas un evento, puedes tener una idea de quién va a asistir porque tienes la lista de participantes cargada con anticipación. Mientras pasa el evento, puedes ir monitoreando el aforo en tiempo real, así sabes si está llegando mucha gente o poca. Cada evento se gestiona de forma independiente, así que no se mezclan los datos. Y lo mejor es que queda todo guardado, así que tienes un historial completo de todos los eventos que hayan hecho, lo cual puede ser útil para análisis posteriores o simplemente para tener registro de todo lo que ha pasado en el departamento.

\subsubsection{Profesionalismo}

Usar este sistema proyecta una imagen más moderna y profesional. Los participantes notan que están usando tecnología actual, no métodos antiguos. Esto mejora su experiencia porque el proceso es rápido y claro. Y para el departamento, facilita mucho la gestión administrativa porque todo está centralizado y organizado.

% Casos de Uso
\section{Casos de Uso Prácticos}

\subsection{Caso 1: Seminario Académico}

\textbf{Escenario:} Imagínense un seminario con 200 participantes esperados. Esto es algo que pasa frecuentemente en el departamento.

\textbf{Con el método tradicional:}
\begin{itemize}
    \item Llegan los participantes y tienes que buscar sus nombres en listas impresas (que a veces se ven mal porque la impresora no estaba bien calibrada)
    \item Vas marcando uno por uno con lápiz, y si cometes un error, tienes que usar goma o tachar
    \item Mientras tanto, no tienes idea de cuántos han llegado realmente. Solo al final cuando cuentas manualmente
    \item Después del evento, tienes que sentarte a pasar todo a Excel, lo cual puede tomar una hora fácil
    \item En total, te toma como 3-4 horas de trabajo solo para la gestión de asistencia
\end{itemize}

\textbf{Con Mi Asistencia:}
\begin{itemize}
    \item Los participantes simplemente ingresan su RUT en un celular o tablet que pones en la entrada
    \item El sistema los registra al instante y les muestra confirmación. Se siente profesional
    \item Mientras pasa el evento, puedes ir viendo en tu propio celular cuántos han llegado, sin tener que interrumpir nada
    \item Al final, descargas el reporte completo en Excel con un solo clic. Tarda menos de un minuto
    \item En total, solo necesitas como 30 minutos para configurar todo antes del evento, y durante el evento prácticamente no haces nada, solo monitoreas
\end{itemize}

\textbf{El resultado:} Ahorras más de 3 horas. Y esas 3 horas las puedes usar en cosas más importantes del evento, o simplemente en descansar después.

\subsection{Caso 2: Capacitación para Funcionarios}

Otro caso común es cuando hacen capacitaciones internas para los funcionarios del departamento. Digamos que son como 50 personas. Con el sistema, al inicio de la sesión todos van pasando por un tablet o celular, ingresan su RUT y ya está. En menos de 10 minutos tienes registrado a todo el mundo.

Lo bueno es que inmediatamente puedes verificar quién asistió. Y si después necesitas enviar la lista a recursos humanos o a quien corresponda, simplemente descargas el reporte y se lo envías. Ya no tienes que estar pasando datos manualmente. Además, queda todo guardado permanentemente, así que si en el futuro necesitas revisar quién asistió a qué capacitación, está ahí guardado.

\subsection{Caso 3: Evento Recurrente}

También pensé en los eventos que se repiten, como los talleres mensuales que hacen. Para estos casos es súper útil porque puedes reutilizar la lista base de participantes. No tienes que estar cargando los mismos nombres una y otra vez. Solo actualizas quién falta o quién se agregó.

Lo interesante es que después puedes comparar la asistencia entre diferentes fechas. Ver si este mes vino más gente que el mes pasado, por ejemplo. O identificar a los participantes que siempre vienen, esos que están más comprometidos. Con esos datos puedes ir mejorando los eventos, sabiendo qué funciona mejor y qué no tanto.

% Características Técnicas (Simplificadas)
\section{Características Técnicas}

\subsection{Seguridad}

La seguridad fue algo que tuve que pensar bien. Para el panel de administración implementé un sistema de login, así que solo las personas autorizadas pueden entrar a gestionar eventos y ver los datos. No cualquiera puede entrar y modificar cosas.

Los datos están guardados en la nube usando Firebase, que es de Google, así que está protegido con sus estándares de seguridad. Toda la información que ingresas pasa por validación, así que si alguien intenta ingresar algo incorrecto, el sistema lo detecta y no lo acepta. Básicamente, solo puedes hacer lo que se supone que debes hacer.

\subsection{Confiabilidad}

Como todo está en la nube, los datos se respaldan automáticamente. No tienes que preocuparte de perder información porque se guarda sola. La aplicación está disponible las 24 horas del día, así que puedes acceder cuando necesites, no hay horarios.

Lo bueno es que no hay riesgo de perder datos. Cada vez que alguien marca asistencia o haces algún cambio, se guarda al instante. Y tienes el historial completo de todos los eventos que hayas hecho, así que si necesitas revisar algo de hace meses, está ahí disponible.

\subsection{Escalabilidad}

No hay límites en cuanto a cuántos participantes puede manejar. He probado con eventos pequeños de 10 personas y también simulé eventos grandes de 1000, y funciona igual de rápido. Puedes tener tantos eventos como necesites en el sistema, no hay restricciones. Y si en el futuro necesitas agregar alguna funcionalidad nueva o cambiar algo, se puede hacer sin problemas.

\subsection{Mantenimiento}

Lo bueno es que no necesitas tener servidores propios ni nada complicado. Todo está en la nube, así que no hay mantenimiento local que hacer. Cuando hago mejoras o correcciones, se actualizan automáticamente sin que tengan que hacer nada ustedes. Si en algún momento tienen algún problema o duda, pueden contactarme y los ayudo. Y lo mejor es que no tienen que instalar nada en los computadores, funciona directo desde el navegador.

% Facilidad de Uso
\section{Facilidad de Uso}

\subsection{Para Registradores de Asistencia}

El proceso es realmente simple. La persona solo tiene que abrir la página en su navegador (cualquiera funciona), ingresar su RUT en el campo que aparece, hacer clic en buscar, y listo. El sistema le muestra una confirmación indicando que su asistencia quedó registrada. No necesita saber nada más, es bastante obvio qué tiene que hacer. No requiere capacitación porque la interfaz es intuitiva, cualquiera puede usarlo sin explicación previa.

\subsection{Para Administradores}

El panel de administración lo diseñé pensando en que fuera fácil de usar. Intenté organizarlo de forma lógica:

\begin{itemize}
    \item \textbf{Pestañas claras:} Cada cosa está en su lugar. Gestión de eventos en una pestaña, participantes en otra, así no te pierdes
    \item \textbf{Iconos y colores:} Usé iconos para que sea más visual. No tienes que leer mucho texto para saber qué hace cada botón
    \item \textbf{Mensajes útiles:} El sistema te dice claramente qué pasó cuando haces algo. Si algo salió mal, te explica qué fue
    \item \textbf{Todo tiene su lugar:} Intenté que no tengas que adivinar dónde está cada función. Si buscas algo, probablemente está donde lo esperarías encontrar
\end{itemize}

\subsection{Capacitación}

La verdad es que no necesita mucha capacitación. Mi objetivo era que fuera intuitivo:

\begin{itemize}
    \item \textbf{Sesión inicial:} Con 30 minutos de explicación ya estás listo para usarlo. La mayoría de las cosas son bastante obvias una vez que las ves
    \item \textbf{Documentación:} Dejé un manual de usuario que explica todo paso a paso, por si acaso olvidas algo
    \item \textbf{Soporte:} Si en algún momento no entiendes algo o algo no funciona como esperas, puedes contactarme y te ayudo
    \item \textbf{Puedes empezar simple:} No necesitas aprender todo de una vez. Puedes empezar solo registrando asistencia, y después cuando te sientas cómodo, exploras las otras funciones
\end{itemize}

% Impacto y Resultados
\section{Impacto y Resultados Esperados}

\subsection{Mejora en Eficiencia}

Basándome en las pruebas que hice y comparándolo con el método tradicional, espero que vean estos resultados:

\begin{itemize}
    \item \textbf{Menos tiempo:} Calculo que ahorran entre 70-80\% del tiempo que normalmente gastarían. En lugar de horas, hablamos de minutos
    \item \textbf{Mucho más preciso:} Casi se eliminan los errores porque el sistema valida todo automáticamente. Si hay un error, es porque el dato original estaba mal, no por transcripción
    \item \textbf{Más productividad:} El tiempo que ahorren lo pueden usar en cosas más importantes del evento, o simplemente en tener menos estrés
    \item \textbf{Todos contentos:} Los administradores tienen menos trabajo, y los participantes no tienen que esperar en filas largas para registrarse
\end{itemize}

\subsection{Mejora en la Calidad de la Información}

Los datos que obtienes son mucho más confiables que cuando lo haces manualmente. Todo está validado automáticamente, así que sabes que la información es correcta. Las estadísticas están disponibles al instante, no tienes que esperar a procesar nada. Y cuando necesitas hacer un análisis más profundo, simplemente exportas los datos a Excel y ya tienes todo listo para trabajar. Lo mejor de todo es que queda guardado permanentemente, así que tienes un registro completo de todos los eventos que puedas consultar cuando quieras.

\subsection{Mejora en la Experiencia del Usuario}

Para los participantes también es mucho mejor. No tienen que estar esperando en filas largas para que alguien busque su nombre en una lista. El proceso es rápido, ingresan su RUT y listo. Inmediatamente ven una confirmación en pantalla que les dice que quedaron registrados, así que no hay dudas. Y pueden usar su propio celular si quieren, no tienen que esperar su turno en un dispositivo compartido. Todo esto les da la impresión de que están en un evento bien organizado y profesional.

\subsection{Rendimiento de la Inversión}

Al final, el sistema se paga solo con el tiempo que ahorra. Cada hora de trabajo administrativo que recuperas es tiempo que puedes usar en otras cosas. Ya no tienes que corregir errores después ni rehacer trabajos porque algo salió mal. Y cuando planificas eventos futuros, ya tienes datos reales de eventos anteriores que te ayudan a tomar mejores decisiones. Es como tener una base sólida sobre la cual seguir construyendo y mejorando.

% Implementación
\section{Implementación y Uso}

\subsection{Requisitos}

Para usar el sistema, las cosas que necesitas son bastante básicas:

\begin{itemize}
    \item \textbf{Internet:} Cualquier conexión funciona, puede ser WiFi o datos del celular
    \item \textbf{Un navegador:} Chrome, Firefox, Safari, Edge, cualquiera de estos funciona bien
    \item \textbf{Un dispositivo:} Cualquiera que tenga navegador. He probado en computadores, tablets y celulares, y todo funciona
    \item \textbf{El link:} Solo necesitas la dirección web que te doy. La abres y ya está
\end{itemize}

\subsection{Proceso de Implementación}

El proceso es bastante simple y rápido:

\begin{enumerate}
    \item \textbf{Primero configuramos:} Nos sentamos juntos y creamos el primer evento, cargamos una lista de prueba. Esto toma como media hora
    \item \textbf{Capacitación rápida:} Te explico cómo funciona todo, que como dije, es bastante intuitivo. Otra media hora
    \item \textbf{Probamos con un evento real:} Lo usamos en un evento real pero pequeño, para que todos se familiaricen sin presión
    \item \textbf{Y listo:} Después de eso, ya lo pueden usar en todos los eventos que quieran
\end{enumerate}

\subsection{Soporte Continuo}

No los voy a dejar solos después de implementarlo. Dejé documentación completa con manuales que explican paso a paso cómo hacer cada cosa. Si en algún momento tienen algún problema o algo no funciona como esperan, pueden contactarme y los ayudo a resolverlo. El sistema lo voy siguiendo mejorando, así que si se me ocurre alguna funcionalidad nueva o si tienen alguna sugerencia, la puedo agregar. Y si necesitan más capacitación o alguna sesión adicional para que todos se sientan más cómodos, no hay problema, con gusto lo hacemos.

% Conclusiones
\section{Conclusiones}

En resumen, \textbf{Mi Asistencia} es algo que desarrollé pensando específicamente en lo que necesitamos en el Departamento de Informática. No es una solución genérica que adapté, sino que la hice desde cero con nuestros problemas en mente.

Creo que esta herramienta puede hacer una diferencia real en cómo manejamos los eventos. No solo resuelve los problemas que todos conocemos, sino que además nos da información que antes simplemente no teníamos. Al empezar a usarla, el departamento va a notar que todo funciona más eficientemente, que tienen información precisa disponible al momento, que el proceso es simple y fácil de usar. Además, proyecta una imagen más profesional y moderna. Y por supuesto, van a ahorrar mucho tiempo y recursos.

El sistema está listo para usarse ahora mismo. Pueden implementarlo de inmediato y empezar a ver los beneficios desde el primer evento que hagan.

\subsection{Próximos Pasos}

Ahora, sobre qué sigue. Primero, espero que revisen esta presentación y me digan qué les parece. Si les interesa, podemos programar una demostración práctica donde les muestro cómo funciona en vivo. Después podríamos definir algún evento pequeño como piloto, algo donde todos se puedan familiarizar sin presión. Una vez que se sientan cómodos, hacemos la capacitación formal para todo el personal que vaya a usar el sistema. Y finalmente, empezamos a usarlo en todos los eventos del departamento.

\vspace{1cm}

Estoy listo para ayudar a transformar la forma en que gestionan la asistencia en el Departamento de Informática. Creo que este sistema puede hacer una diferencia real y estoy entusiasmado con la posibilidad de verlo en acción.

% Contacto
\newpage
\section*{Información de Contacto}

Si tienen preguntas, quieren ver una demostración o están listos para implementarlo, pueden contactarme. El sistema se llama Mi Asistencia, está desarrollado para el Departamento de Informática de Santo Tomás Temuco, y yo soy el desarrollador, Gerson Valdebenito.

\vspace{0.5cm}

Espero que esta presentación les haya dado una buena idea de lo que puede hacer este sistema. Lo desarrollé pensando específicamente en las necesidades que tenemos en el departamento, así que creo que puede ser realmente útil para todos.

% Anexos
\newpage
\appendix
\section{Anexo A: Capturas de Pantalla del Sistema}

\subsection{Interfaz Principal de Registro de Asistencia}

La siguiente imagen muestra la interfaz principal donde los participantes ingresan su RUT para registrar asistencia. El diseño es limpio, centrado y fácil de usar.

\textit{Nota: Se recomienda agregar captura de pantalla de la página principal de registro de asistencia aquí.}

\subsection{Confirmación de Asistencia}

Cuando un participante marca su asistencia, el sistema muestra inmediatamente una confirmación con sus datos, verificando que el registro fue exitoso.

\textit{Nota: Se recomienda agregar captura de pantalla de la confirmación de asistencia aquí.}

\subsection{Panel de Administración}

El panel de administración permite gestionar eventos, participantes y ver estadísticas. La interfaz está organizada en pestañas para facilitar la navegación.

\textit{Nota: Se recomienda agregar captura de pantalla del panel de administración aquí.}

\subsection{Estadísticas en Tiempo Real}

Las estadísticas se actualizan automáticamente mostrando totales, presentes y ausentes. Permiten exportar esta información a Excel con un solo clic.

\textit{Nota: Se recomienda agregar captura de pantalla de las estadísticas aquí.}

\subsection{Gestión de Eventos}

La sección de gestión de eventos permite crear, editar y activar eventos fácilmente. Se pueden filtrar por tipo (alumnos o funcionarios).

\textit{Nota: Se recomienda agregar captura de pantalla de la gestión de eventos aquí.}

\subsection{Vista Responsive en Dispositivo Móvil}

El sistema se adapta perfectamente a dispositivos móviles, permitiendo el registro de asistencia desde cualquier celular o tablet.

\textit{Nota: Se recomienda agregar captura de pantalla del sistema en móvil aquí.}

\section{Anexo B: Manual de Usuario Básico}

\subsection{Registro de Asistencia (Usuario Final)}

\subsubsection{Paso 1: Acceder al Sistema}
\begin{enumerate}
    \item Abrir un navegador web (Chrome, Firefox, Safari, Edge)
    \item Ingresar la URL proporcionada por el administrador
    \item La página principal se cargará automáticamente
\end{enumerate}

\subsubsection{Paso 2: Ingresar RUT}
\begin{enumerate}
    \item En el campo "RUT", ingresar el número de RUT del participante
    \item El sistema acepta RUT con o sin puntos y guión
    \item Hacer clic en el botón "Buscar"
\end{enumerate}

\subsubsection{Paso 3: Confirmar Asistencia}
\begin{enumerate}
    \item Si el participante está registrado, se mostrarán sus datos
    \item Se mostrará un mensaje de confirmación: "Asistencia Realizada"
    \item Los datos del participante se mostrarán en pantalla
    \item El proceso está completo y se puede registrar otro participante
\end{enumerate}

\subsection{Gestión de Eventos (Administrador)}

\subsubsection{Crear un Nuevo Evento}
\begin{enumerate}
    \item Ingresar al panel de administración
    \item Ir a la pestaña "Gestión de Eventos"
    \item Hacer clic en "Crear Evento"
    \item Completar el formulario:
    \begin{itemize}
        \item Nombre del evento
        \item Descripción
        \item Fecha de inicio
        \item Fecha de fin
        \item Tipo (Alumnos o Funcionarios)
    \end{itemize}
    \item Hacer clic en "Crear"
\end{enumerate}

\subsubsection{Activar un Evento}
\begin{enumerate}
    \item En la lista de eventos, encontrar el evento deseado
    \item Hacer clic en "Activar"
    \item El evento quedará activo y disponible para registro de asistencia
    \item Solo puede haber un evento activo a la vez
\end{enumerate}

\subsubsection{Importar Lista de Participantes}
\begin{enumerate}
    \item Ir a la pestaña de participantes en el panel de administración
    \item Hacer clic en "Importar"
    \item Seleccionar archivo Excel o JSON con la lista
    \item El sistema validará los datos
    \item Confirmar la importación
\end{enumerate}

\subsection{Exportar Reportes}

\subsubsection{Exportar Lista Completa}
\begin{enumerate}
    \item En el panel de administración, ir a la lista de participantes
    \item Hacer clic en "Exportar a Excel"
    \item Se descargará un archivo Excel con todos los datos
\end{enumerate}

\subsubsection{Exportar Estadísticas Filtradas}
\begin{enumerate}
    \item En la sección de estadísticas
    \item Seleccionar el filtro deseado (Presentes, Ausentes, Total)
    \item Hacer clic en el ícono de Excel en la tarjeta correspondiente
    \item Se descargará el reporte filtrado
\end{enumerate}

\section{Anexo C: Flujos de Trabajo}

\subsection{Flujo: Registro de Asistencia en un Evento}

\begin{enumerate}
    \item \textbf{Preparación del Evento} (Administrador)
    
    Primero se crea el evento en el sistema, que es bastante simple. Solo pones el nombre, las fechas y el tipo. Después cargas la lista de participantes, que puedes hacer de dos formas: importar desde Excel si ya tienes una lista, o agregar personas de a una si prefieres. Una vez que todo está listo, activas el evento y ya está disponible para que la gente se registre.
    
    \item \textbf{Durante el Evento} (Registradores)
    
    Durante el evento, los participantes abren la aplicación en cualquier dispositivo que tengan disponible. Pueden usar tablets, celulares, computadores, lo que sea más cómodo. Ingresan su RUT, el sistema los busca automáticamente y registra su asistencia. Les aparece una confirmación inmediata en pantalla, así que saben que quedó registrado. Todo esto pasa en segundos.
    
    \item \textbf{Monitoreo} (Administrador)
    
    Mientras el evento está pasando, el administrador puede estar viendo las estadísticas en tiempo real. Puede monitorear cómo va progresando la asistencia, ver cuántos han llegado, cuántos faltan. Todo se actualiza automáticamente, así que no tiene que estar recargando la página constantemente.
    
    \item \textbf{Finalización} (Administrador)
    
    Al finalizar el evento, el administrador descarga el reporte final en Excel con un solo clic. Desactiva el evento para que no siga recibiendo registros, y toda la información queda archivada en el sistema para futuras referencias. Todo esto tarda literalmente minutos.
\end{enumerate}

\subsection{Flujo: Gestión de Evento Recurrente}

Para eventos que se repiten periódicamente, como los talleres mensuales que hacen, el proceso es aún más simple. Creas el evento con la fecha específica del mes. Como ya tienes la lista base de participantes guardada, simplemente la reutilizas. Si hay que agregar o quitar alguien, lo haces rápido. Cuando llega la fecha, activas el evento y listo.

Después del evento, puedes comparar la asistencia con los meses anteriores. Ves si este mes vino más o menos gente, quiénes son los participantes más constantes, esas cosas. Con esa información puedes ir mejorando los eventos futuros, sabiendo qué funciona y qué no tanto.

\section{Anexo D: Comparativa Antes y Después}

\subsection{Comparativa de Tiempo}

\begin{table}[H]
\centering
\begin{tabular}{|p{6cm}|p{4cm}|p{4cm}|}
\hline
\textbf{Actividad} & \textbf{Método Tradicional} & \textbf{Con Mi Asistencia} \\
\hline
Preparación de listas & 30-60 minutos & 15 minutos (importación) \\
\hline
Registro durante evento (200 personas) & 2-3 horas & 30 minutos \\
\hline
Conteo manual & 30 minutos & Instantáneo \\
\hline
Transcripción a Excel & 1 hora & 1 minuto (exportación) \\
\hline
Generación de reportes & 30 minutos & Instantáneo \\
\hline
\textbf{TOTAL} & \textbf{4-5 horas} & \textbf{45 minutos} \\
\hline
\textbf{AHORRO} & - & \textbf{3.5-4 horas} \\
\hline
\end{tabular}
\caption{Comparativa de tiempo invertido en gestión de asistencia}
\end{table}

\subsection{Comparativa de Errores}

\begin{table}[H]
\centering
\begin{tabular}{|p{6cm}|p{4cm}|p{4cm}|}
\hline
\textbf{Tipo de Error} & \textbf{Método Tradicional} & \textbf{Con Mi Asistencia} \\
\hline
RUT incorrecto por transcripción & Frecuente (5-10\%) & No aplica (validación automática) \\
\hline
Marcas duplicadas & Ocasional (2-5\%) & No aplica (sistema previene duplicados) \\
\hline
Pérdida de información & Ocasional (1-2\%) & No aplica (almacenamiento automático) \\
\hline
Errores en conteo & Frecuente (3-7\%) & No aplica (conteo automático) \\
\hline
\textbf{TOTAL DE ERRORES} & \textbf{11-24\%} & \textbf{0\%} \\
\hline
\end{tabular}
\caption{Comparativa de errores en el proceso}
\end{table}

\subsection{Comparativa de Recursos Necesarios}

\begin{itemize}
    \item \textbf{Método Tradicional:}
    \begin{itemize}
        \item Personal: 2-3 personas para registro
        \item Materiales: Listas impresas, lápices, gomas
        \item Equipos: Computador para transcripción posterior
        \item Tiempo de personal: 4-5 horas
    \end{itemize}
    
    \item \textbf{Con Mi Asistencia:}
    \begin{itemize}
        \item Personal: 1 persona supervisando
        \item Materiales: Ninguno (todo digital)
        \item Equipos: Tablets o celulares (pueden ser personales)
        \item Tiempo de personal: 45 minutos (configuración inicial)
    \end{itemize}
\end{itemize}

\section{Anexo E: Preguntas Frecuentes}

\subsection{¿Necesito instalar algo en mi computador?}
No, para nada. El sistema funciona completamente en el navegador web, así que no tienes que instalar nada. Solo necesitas tener internet y abrir cualquier navegador, ya sea Chrome, Firefox, Safari, el que uses normalmente.

\subsection{¿Funciona en celulares y tablets?}
Sí, funciona perfectamente. Probé que funcionara bien en celulares, tablets y computadores porque nunca sabes qué dispositivo va a usar cada persona. Así que da lo mismo si alguien llega con su celular o si tienes una tablet en la entrada, todo funciona igual.

\subsection{¿Cuántas personas pueden registrar asistencia al mismo tiempo?}
No hay límite. Pueden estar varias personas registrando asistencia al mismo tiempo desde diferentes dispositivos. Es súper útil cuando hay mucha gente llegando a la vez, porque puedes poner varios celulares o tablets en diferentes puntos y todos trabajan simultáneamente.

\subsection{¿Qué pasa si no hay internet durante el evento?}
El sistema necesita internet para funcionar, así que es importante verificar que haya conexión antes del evento. Si por alguna razón se cae el internet durante el evento, cuando se restablezca la conexión el sistema guarda automáticamente todo lo que se había registrado. Pero lo ideal es tener una conexión estable desde el principio.

\subsection{¿Los datos están seguros?}
Sí, están seguros. Los datos están guardados en la nube usando Firebase, que es de Google, así que cumple con todos los estándares de seguridad que usan empresas grandes. Solo las personas que tienen credenciales de administrador pueden acceder a la información, así que no cualquiera puede entrar y ver los datos.

\subsection{¿Puedo exportar los datos después del evento?}
Claro, sí puedes. Puedes exportar las listas completas o si solo quieres ver quiénes llegaron o quiénes faltaron, puedes filtrar antes de exportar. Y lo puedes hacer en cualquier momento, no solo durante el evento. Si después de una semana necesitas el reporte, simplemente lo descargas.

\subsection{¿Cuántos eventos puedo crear?}
No hay límite. Puedes crear todos los eventos que necesites y gestionarlos todos desde el panel de administración. Cada uno es independiente, así que no se mezclan los datos entre eventos.

\subsection{¿Qué pasa con los datos después del evento?}
Quedan guardados permanentemente. No se borran, así que puedes consultarlos cuando quieras, exportarlos si los necesitas, o incluso reactivar el evento más adelante si por alguna razón necesitas seguir registrando asistencia. Todo queda ahí disponible.

\subsection{¿Necesito capacitación especial?}
No realmente. El sistema es bastante intuitivo, así que la mayoría de las personas pueden empezar a usarlo sin explicación. Pero sí hago una sesión inicial de como media hora para que todos se familiaricen con las funciones más avanzadas, aunque honestamente el uso básico lo puedes hacer desde el primer día.

\subsection{¿Puedo personalizar el sistema para nuestras necesidades?}
Sí, claro. El sistema ya está diseñado pensando en las necesidades del Departamento de Informática, pero si en el futuro necesitan alguna funcionalidad adicional o algún cambio específico, podemos hablar y ver cómo implementarlo. No es un sistema cerrado, se puede ajustar según lo que necesiten.

\section{Anexo F: Glosario de Términos}

\begin{description}
    \item[Evento Activo] Evento que está actualmente habilitado para recibir registros de asistencia. Solo puede haber un evento activo a la vez.
    
    \item[Panel de Administración] Sección del sistema donde los administradores pueden gestionar eventos, participantes y ver estadísticas.
    
    \item[Registro en Tiempo Real] Actualización inmediata de información sin necesidad de recargar la página.
    
    \item[Importación Masiva] Proceso de cargar múltiples participantes al mismo tiempo desde un archivo Excel o JSON.
    
    \item[Exportación Filtrada] Descarga de datos aplicando filtros específicos (solo presentes, solo ausentes, etc.).
    
    \item[Estadísticas en Vivo] Información numérica que se actualiza automáticamente conforme se registran asistencias.
\end{description}

\section{Anexo G: Requisitos del Sistema}

\subsection{Requisitos Mínimos para el Usuario}

\begin{itemize}
    \item \textbf{Dispositivo:} Cualquier dispositivo con navegador web
    \begin{itemize}
        \item Computador (Windows, Mac, Linux)
        \item Tablet (Android, iPad)
        \item Celular (Android, iPhone)
    \end{itemize}
    
    \item \textbf{Navegador:} Cualquier navegador moderno
    \begin{itemize}
        \item Google Chrome (recomendado)
        \item Mozilla Firefox
        \item Safari
        \item Microsoft Edge
        \item Opera
    \end{itemize}
    
    \item \textbf{Conexión:} Internet estable (puede ser WiFi o datos móviles)
    
    \item \textbf{Permisos:} Ninguno especial, solo acceso a la URL proporcionada
\end{itemize}

\subsection{Requisitos para Administradores}

\begin{itemize}
    \item Todo lo anterior, más:
    \item \textbf{Cuenta de administrador:} Credenciales proporcionadas por el desarrollador
    \item \textbf{Capacitación básica:} Sesión de 30 minutos (opcional pero recomendada)
\end{itemize}

\section{Anexo H: Contacto y Soporte}

\subsection{Información de Contacto}

Para soporte técnico, consultas o implementación:

\begin{itemize}
    \item \textbf{Sistema:} Mi Asistencia
    \item \textbf{Departamento:} Informática - Santo Tomás Temuco
    \item \textbf{Desarrollador:} Gerson Valdebenito
\end{itemize}

\subsection{Tipos de Soporte Disponible}

\begin{itemize}
    \item \textbf{Capacitación inicial:} Sesión de capacitación para administradores
    \item \textbf{Soporte durante implementación:} Asistencia en los primeros eventos
    \item \textbf{Documentación:} Manuales y guías disponibles
    \item \textbf{Soporte técnico:} Resolución de problemas cuando se presenten
\end{itemize}

\subsection{Recursos Adicionales}

\begin{itemize}
    \item Manual de usuario completo
    \item Guías rápidas por funcionalidad
    \item Video tutoriales (si están disponibles)
    \item FAQ actualizado
\end{itemize}

\end{document}

